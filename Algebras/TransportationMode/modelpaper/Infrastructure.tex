\section{Representation for Infrastructures}
\label{sec:RfI}
In this section, we give the location representation for each infrastructure. Section \ref{publictn}
introduces the public transportation network and Section \ref{sec:indoor} addresses the \textit{indoor} environment. Region-based outdoor is discussed in Section \ref{regionbased}. Road network and free space are described in Sections \ref{sec:roadnetwork} and \ref{sec:freespace}, respectively. 

\subsection{Public Transportation Network}
\label{publictn}
The elements (\textit{infrastructure objects}) in this kind of infrastructure include buses, trains, and underground trains. 
As they have similar characteristics, without loss of generality we take the public bus as an example to present how the infrastructure and location is represented. The bus network locates in the environment \textit{road network} but with additional features. It has two types of components: 
\textit{static} and \textit{dynamic}, where the first includes $bus$ $stops$ and $bus$ $routes$, 
and the second is moving buses. Each bus has a regular mobility pattern as 
it normally moves along a bus route which is a pre-defined line in a road network and has a sequence of bus stops where the movement suspends. 
\subsubsection{Static}
\label{sec:static}
A $bus$ $stop$ identifies a location in space where a bus first arrives,
lets passengers get on and off, and then departs (if it is not the last stop). 
Each $bus$ $stop$ belongs to one bus route, but several bus stops from different routes 
can have the same spatial location where the transfer occurs. A $bus$ $route$ contains a sequence of 
$bus$ $stops$ which partition the whole route into a sequence of 
sub lines, each part is called a \textit{route segment} (\textit{segment} for short). 
Each segment defines the connection between 
two consecutive stops. The bus can only stop at these places for passengers getting
on and off. A route has a start location and an end location which can be equal to each other 
(a cycle route). The route may be not the shortest path connecting two locations in a
road network, but it defines a path along which the bus should move. \\

\textbf{Bus Stop} \\

Intuitively, a bus stop corresponds to a point in space, but it also has some other 
information, e.g., which bus route it belongs to. Instead of simply using a point to identify 
a bus stop, we define a data type named $\underline{busstop}$ with its carrier set given
in the following. 
 
\begin{Statement}
\label{defintionbusstop}
 Bus Stop 

$D_{\underline{busstop}}$ = \{$(rid, pos)|rid, pos \in int, rid\geq 0 \wedge pos \geq 0$\}
\end{Statement}

$rid$ and $pos$ denote which bus route the stop belongs to and the relative order in that route, respectively. We define a partial order ``$\prec$'' on $\underline{busstop}$ objects. Given $bs_1$, $bs_2 \in D_{\underline{busstop}}$, \\

$bs_1 \prec bs_2 \Leftrightarrow bs_1.rid=bs_2.rid \wedge$ $bs_1.pos,bs_2.pos$ are \textit{consequent}  \\

If two bus stops fulfill the condition ``$\prec$'', they are called \textit{consequent}. 
Let $p\_bs_i(\in D_{\underline{point}})$ be the spatial point $bs_i(\in D_{\underline{busstop}})$ corresponds to (later, we introduce how to retrieve it). Then, the equality of two bus stops is defined as:  \\

$bs_1=bs_2\Leftrightarrow$ $p\_bs_1$ = $p\_bs_2$. \\

There are two possibilities if $bs_1=bs_2$: \\

(1) $bs_1.rid=bs_2.rid \wedge bs_1.pos=bs_2.pos$; \\

(2) $bs_1.rid\neq bs_2.rid$. \\

The first case should be straightforward, while the second illustrates that two 
different bus routes can have an intersection bus stop where the \textit{bus transfer} can happen. 
The geo-location of a bus stop is described by a point, but a point
can correspond to several bus stops belonging to different bus routes. \\

\textbf{Bus Route} \\

A bus route consists of a sequence of \textit{segments}, each of which represents the geographic 
connection between two consecutive bus stops. Consequently, we represent a bus route 
by a sequence of segments. Let $Busseg$ be the set of bus segments defined as below:  

\begin{Statement}
\label{defintionseg}
 Bus Segment 

$Busseg$ = 
\{$(bs_1,bs_2,geo)|bs_1,bs_2 \in D_{\underline{busstop}},bs_1 \prec bs_2$, 
$geo \in D_{\underline{line}}$\}
\end{Statement}

Each segment is composed of two bus stops recording two endpoints and a line $geo$ denoting 
the geographical connection in space. It represents the connection between 
two \textit{consequent} bus stops. Given two segments $seg_1$, $seg_2 \in Busseg$, we define\\

$seg_1$ is $linkable$ to $seg_2\Leftrightarrow seg_1.bs_2=seg_2.bs_1$, \\

that is $seg_1.bs_2$ and $seg_2.bs_1$ have the same spatial point in space. \\

Let $\underline{busroute}$ be the data type for \textit{bus routes} and the domain is: 

\begin{Statement}
\label{defintionbusroute}
 Bus Route 

$D_{\underline{busroute}}$ = 
\{$<seg_1$,$seg_2$,...,$seg_n> | n\geq1, n \in D_{\underline{int}}$, and 

(1)$\forall i \in [1,n], seg_i \in Busseg$;

(2) $\forall i \in[1,n$-$1], seg_i.bs_1.rid$=$seg_{i+1}.bs_1.rid 
\wedge seg_i$ is $linkable$ to $seg_{i+1}$ \}
\end{Statement}

A \textit{bus route} is defined as a sequence of bus segments where 
all of them have the same $rid$ and each two consequent segments are $linkable$. 
Let operator \textbf{geo\_data} return the geometry line of a bus
segment and the whole line of a bus route can be constructed via
$\bigcup_{i=1}^{n}$\textbf{geo\_data}($seg_i$). Two consecutive segments
$seg_i$ and $seg_{i+1}$ have an intersection point in space where $seg_i$ ends at it and 
$seg_{i+1}$ starts from it, and the point identifies the position of a 
bus stop. Given a bus route $r$, let $m$ be the number of its segments and $n$ be the 
number of bus stops. Then, $m$ and $n$ satisfy

\begin{equation}
\label{stopandsegment}
 r.n=r.m+1. 
\end{equation}

And given a $\underline{busstop}$ $bs=<rid,pos>$, the indices for 
the bus segments ending and starting at it are $bs.pos$-$1$, $bs.pos$. 
Thus, the geometry data (a $point$) for the bus stop can be obtained which is the endpoint of a bus segment. \\

To illustrate how the above data types work together, Figure \ref{fig:sbusexample} 
gives an example with three bus routes $\{r_1, r_2, r_3\}$ (depicted by 
different line styles: solid, dashed and dotted) and 
five points $\{s_1, s_2, s_3, s_4, s_5\}$ to identify bus stop locations. The points
are connected by six segments from different bus routes (labeled by symbol $seg$ with a
subscript) where $seg_1$, $seg_2$ are for $r_1$, $seg_3$, 
$seg_4$ are for $r_2$, and $seg_5$, $seg_6$ are for $r_3$. Points 
$s_2$ and $s_3$ are connected by two segments $seg_2$ and $seg_3$. They are treated as 
different bus segments as they belong to 
different bus routes and can have distinct geometry lines in space. 
Figure \ref{tab:busrouteandstop} lists the representation for bus stops 
and bus routes. For brevity, we omit the detailed geometry description for points and lines
here. The points from \{$s_1$,$s_2$,$s_3$,$s_4$\} represent the intersection 
locations of two bus routes where each denotes the position of two bus stops. 
It implies that a \textit{bus transfer} can happen here. For example, at point $s_2$ one can
switch from route $r_1$ to $r_2$. Only point $s_5$ corresponds to one bus stop as no bus 
routes intersect here. 

\begin{figure}[htb]
%\centering
% \includegraphics[scale=0.1,width=2.8in,height=1.8in]{images/busexample.eps}
\psset{unit=0.7mm}
	\centering
	\begin{pspicture}(0,0)(70,60)
	\cnodeput(10,50){A}{\footnotesize{$s_1$}}	
	\cnodeput(6,25){B}{\footnotesize{$s_2$}}
	\cnodeput(13,2){C}{\footnotesize{$s_3$}}		
	\cnodeput(40,28){D}{\footnotesize{$s_4$}}
	\cnodeput(65,25){E}{\footnotesize{$s_5$}}		
	\ncarc[arcangle=-40]{->}{A}{B}
	\nbput{\footnotesize{$seg_1$}}
	\ncarc[arcangle=-40]{->}{B}{C}
	\nbput{\footnotesize{$seg_2$}}
	\ncarc[arcangle=40,linestyle=dashed]{->}{B}{C}
	\naput{\footnotesize{$seg_3$}}
	\ncarc[arcangle=-40,linestyle=dashed]{->}{C}{D}
	\nbput{\footnotesize{$seg_4$}}
	\ncarc[arcangle=20,linestyle=dotted]{->}{A}{D}
	\naput{\footnotesize{$seg_5$}}
	\ncarc[arcangle=20,linestyle=dotted]{->}{D}{E}
	\naput{\footnotesize{$seg_6$}}
	\psline{->}(50,60)(65,60)
	\rput(70,60){\footnotesize{$r_1$}}
	\psline[linestyle=dashed]{->}(50,55)(65,55)
	\rput(70,55){\footnotesize{$r_2$}}
	\psline[linestyle=dotted]{->}(50,50)(65,50)
	\rput(70,50){\footnotesize{$r_3$}}

	\end{pspicture}
	\caption{\label{fig:sbusexample} A Simple Bus Network} 
 \end{figure}

\begin{figure}[htb]
\centering
\subfigure
  {\begin{tabular}{c|c}
	\hline
    Point& Bus Stops\\
	\hline
	$s_1$& $bs_1=<1,1>$, $bs_7=<3,1>$\\
	\hline
	$s_2$&  $bs_2=<1,2>$, $bs_4=<2,1>$\\
	\hline
	$s_3$ & $bs_3=<1,3>$, $bs_5=<2,2>$\\
	\hline
	$s_4$ & $bs_6=<2,3>$, $bs_8=<3,2>$\\
	\hline
	$s_5$ & $bs_9=<3,3>$\\
	\hline
  \end{tabular}
}
\hspace{1cm}
\subfigure
  {\begin{tabular}{c|c}
	\hline
    RouteId & Bus Routes\\
	\hline
	1 & $br_1=\{seg_1(bs_1,bs_2)$, $seg_2(bs_2,bs_3)\}$\\
	\hline
	2 & $br_2=\{seg_3(bs_4,bs_5)$, $seg_4(bs_5,bs_6)\}$\\
	\hline
	3 & $br_3=\{seg_5(bs_7,bs_8)$, $seg_6(bs_8,bs_9)\}$\\
	\hline
  \end{tabular}
}
 \caption{\label{tab:busrouteandstop}Static Component}
\end{figure}

One issue that needs to be discussed is in the real world a bus stop is combined with a name and 
normally it has two different positions in the road where each corresponds to one direction 
along the route (\textit{Up} and \textit{Down}). 
In fact, the two stops belong to one bus route but with 
different directions. This invokes the question how to distinguish between them. 
To solve the issue, we define each bus route with two directions as two 
different routes where each is uniquely identified and represented by different geomerical lines.
In the real world, the \textit{Up} and \textit{Down} routes are normally located on two different
lanes of a road and the geometry description for them should be different. 

\subsubsection{Dynamic}
\label{sec:dynamic}
A moving bus contains two types of information: $bus$ $route$ and $schedule$. The route
is fixed and limits the bus movement. The buses belonging to the same 
route have different time schedules, i.e., \textit{departure} and \textit{arrival} time at each stop. 
Each schedule for a bus is called a \textit{bus trip}. 
We model each \textit{bus trip} as a moving point and represent
its movement as follows. Let $Bustu$ be the domain of temporal units for a \textit{bus trip}, which is defined in the following. 

\begin{Statement}
\label{defintiontempunitbus}
Temporal Units for a Bus Trip 

$Bustu = \{(i,bs_1,bs_2)| i \in D_{\underline{interval}}$, 
$bs_1,bs_2 \in D_{\underline{busstop}}$, and  

(i) $i.s = i.e \Rightarrow bs_1 = bs_2$;

(ii)  $bs_1 \prec bs_2$ $\vee$ $(bs_1.rid$ = $bs_2.rid$ $\wedge$ $bs_1.pos$=$bs_2.pos)$ \}
\end{Statement}

The values have three components where the first defines a time interval 
and the second and third stand for two bus stops. The above definition is derived from Definition 
\ref{generaltemporalunit} where $R$ is specified as $D_{\underline{busstop}}\times$
$D_{\underline{busstop}} $. Each element in $Bustu$ describes that at time 
instant $i.s$ the location is identified by $bs_1$, 
and at time instant $i.e$ the object moves to $bs_2$. Note that condition (ii) includes one
more case that values of $rid$ and $pos$ for two bus stops can be the same besides two bus stops fulfilling the condition ``$\prec$''. In the following, we elaborate its meaning. According to the value $rid$ and $pos$ in $bs_1$ and $bs_2$ during time interval $i$, 
a unit $tub \in Bustu$ represents two cases: \\

 \textbf{case 1}:  $bs_1,bs_2$ are \textit{consequent}\\

 \textbf{case 2}: $bs_1,bs_2$ map to the same spatial point in space\\


The first illustrates a normal movement from one bus stop to the successive one where
both of them belong to the same bus route. 
The second denotes a small deviation time at the bus stop illustrating 
the behavior that a bus stays at the stop waiting for passengers getting on 
and off. With the above two cases, the temporal unit can represent different movement behaviors of a bus. \textbf{Case 1} can be considered as the movement in both \textit{temporal} and \textit{spatial} dimension, and \textbf{Case 2} is the movement only in \textit{temporal} dimension. Note that Definition \ref{defintiontempunitbus} applies for the framework of temporal units in Definition \ref{generictemporalunit}. Let $v \in Gentu, u\in Bustu$, and we have \\

(1) $v.i\rightarrow u.i$ 

(2) $v.oid \rightarrow u.bs_1.rid$

(3) $v.i_{loc1} \rightarrow (u.bs_1.pos,\bot)$

(4) $v.i_{loc2} \rightarrow (u.bs_2.pos,\bot)$ 

(5) as $u$ denotes a specified infrastructure unit, $v.m(Bus)$ is not needed in $u$\\


Let $\underline{mpptn}$ be the data type representing \textit{bus trips}. Based on Definition \ref{defintiontempunitbus}, we give the definition as follows.

\begin{Statement}
\label{defintionmovingbus}
Bus Trips  

$D_{\underline{mpptn}}$ = \{$<tub_1$,$tub_2$,...,$tub_n>|n \geq 1,n \in int$, and

(i)$\forall i \in [1,n], tub_i \in Bustu$; 

(ii) $\forall i,j \in[1,n], i \neq j \Rightarrow tub_i.i \cap tub_j.i=\oslash$; 

(iii) $\forall i,j \in [1,n], tub_i.bs_1.rid$=$tub_j.bs_1.rid$; \}

\end{Statement}

Condition (3) is to guarantee that each \textit{bus trip} only
belongs to one bus route so that all \textit{temporal units} have the same route identifier.
For a bus, the regular mobility pattern is that it starts from a bus stop,
goes forward to another, normally staying there for a short time period (e.g., 30 seconds), 
and then proceeds to the next. We define the behavior that the bus stays at a bus stop for a short time 
as a special kind of movement where the spatial location remains constant over time, called $t_{move}$. In contrast to $t_{move}$, the behavior that the bus moves  
from one bus stop to another is called $st_{move}$ where the spatial content changes 
over time. These two movements occur alternatively. 
Instead of explicitly identifying the location by geographical data, the trajectory of a 
moving bus is represented by a sequence of $units$ describing all bus stops it reaches  
in conjunction with the temporal property. Each \textit{unit} is described by a time interval
and two bus stops. The two bus stops can have the 
same or different locations in space corresponding to $t_{move}$ and $st_{move}$, 
respectively. As each bus moves along a bus route represented by a line curve in space,
the position between two stops can be achieved by \textit{linear interpolation} along the geometry of that curve. This representation is more compact because compared with the times of frequently 
updating raw  location data (e.g., coordinates), the number of bus stops is very small. Both update and storage cost are reduced. The units in each bus trip apply for the framework of generic temporal units and bus trips also apply for the framework of generic moving objects in Definition \ref{generictrajectory}. \\


\subsubsection{Infrastructure Objects}
\label{sec:ioinptn}
With the above data types, we give the specific representation for 
\textit{infrastructure objects} in $I_{ptn}$. For static infrastructure objects, we extend Definition \ref{infraobjectsymbol} to include the symbols and types for bus stops and bus routes. \\

\begin{Statement}
\label{infraobjectextend}
\ Extended Infrastructure Objects Data Types and Symbol Data Types 

IOType = IOType $\cup$ \{\underline{busstop}, \underline{busroute}\}

IOSymbol = IOSymbol $\cup$ \{BUSSTOP, BUSROUTE\}

\end{Statement}


Applying Definition \ref{infraobject}, there are two kinds of \textit{infrastructure objects}: 

\begin{itemize}
 \item static: $IO_{ptn}(oid$, \textit{BUSSTOP}, $\beta$, $name)(\beta \in D_{\underline{busstop}})$

 \hspace{0.7cm} $IO_{ptn}(oid$, \textit{BUSROUTE}, $\beta$, $name)(\beta \in D_{\underline{busroute}})$ 

 \item dynamic: $IO_{ptn}(oid$, \textit{MPPTN}, $\beta$, $name)(\beta \in D_{\underline{mpptn}})$

\end{itemize}

The \textit{static} objects are \textit{referenced} by \textit{dynamic} objects, while humans movement is represented by \textit{referencing} to \textit{dynamic infrastructure objects}. 
Applying the framework of \textit{generic temporal units} in Definition \ref{generictemporalunit}, let $u_{ptn}(i,oid,i_{loc_1},i_{loc_2},m)$ be a temporal unit representing the movement of a human in $ptn$ where the concrete attribute values are: \\


(1) $oid\rightarrow IO_{ptn}(oid$, \textit{MPPTN}, $\beta$, $name)(\beta \in D_{\underline{mpptn}})$; 

(2) $i_{loc_1}\rightarrow (\bot,\bot)$; 

(3) $i_{loc_2}\rightarrow (\bot,\bot)$; 

(4) $m\rightarrow Bus$. \\

It represents that during time interval $i$ the object's movement is determined by the \textit{infrastructure object} identified by $oid$. We believe it is not so important to know the passenger's relative position in a bus so that the value $i_{loc_1}$ and $i_{loc_2}$ is set as undefined. The representation can efficiently reduce the
storage size for the traveler's trajectory. Assuming a traveler takes a bus, 
instead of recording the units at all possible locations that the bus speed or direction 
changes, now only one unit is required. And if several passengers take the same bus, all of them
can \textit{reference} to the same \textit{infrastructure object}. Otherwise, the location data for each of them has to be stored. In addition, one does not have to update the data for travelers until they get off the bus or switch to another one. \\


\textbf{Bus Trajectory and Traveller's Trajectory} \\

According to Definition \ref{genrange}, the bus trajectory is a set of $sub\_traj_i(\in Subrange)$
where the attribute values are: \\

(1) $sub\_traj_i.oid \rightarrow IO_{ptn}(oid$, \textit{BUSROUTE}, $\beta$, $name)(\beta \in D_{\underline{busroute}})$;

(2) $sub\_traj_i.l$ stores the movement along the route;

(3) $sub\_traj_i.m \rightarrow Bus$. \\


The elements ($sub\_traj_i \in Subrange$) of a traveller's trajectory projecting to this infrastructure are specified as: \\

(1) $sub\_traj_i.oid \rightarrow IO_{ptn}(oid$, \textit{MPPTN}, $\beta$, $name)(\beta \in D_{\underline{mpptn}})$;

(2) $sub\_traj_i.l \rightarrow \bot$;

(3) $sub\_traj_i.m \rightarrow Bus$. \\

As stated above, we do not describe the passenger's relative position in a bus so that $sub\_traj_i.l$ is set undefined. To get the spatial curve in space, $oid$ is used to get the bus trip and then retrieve the bus trajectory. 

\subsection{Indoor}
\label{sec:indoor}
In the \textit{indoor} environment, there are two kinds of elements where one is for the place that people can stay and move inside such as rooms, chambers, corridors, and the other is for transitions 
between two places, such as office doors, entrances/exits for staircases. 
We call the first $groom$ (general room) and the second $door$. 
Normally, a $groom$ contains at least one door. For example, an office room may have only 
one door, but a corridor can have several doors via which people can leave 
one room, and enter another. The space of a door is covered by the $groom$ it belongs to. 
And a door builds the connection between two $grooms$. 
We represent the position of an \textit{indoor} moving object by mapping it to a $groom$. In Section \ref{sec:modelingindoorspace}, we give the data type representing $groom$ and $door$ as well as the location representation for \textit{indoor} moving objects. In Section \ref{sec:navigation} we create a graph for \textit{indoor} navigation. 
\subsubsection{Modeling Indoor Space}
\label{sec:modelingindoorspace}
\textbf{groom} \\
 
The whole indoor environment comprises of a set of 3D geometry objects where each
corresponds to an office room, a corridor, a chamber, etc. The object can be modeled as a 
3D polygon. Let $Region3d$ be the set of 3D polygons defined as follows: 

\begin{Statement}
\label{3dregion}
\ Region3d 

$Region3d=\{(poly,h)|poly \in D_{\underline{region}},
h \in D_{\underline{real}}\}$
\end{Statement}

The attribute $poly$ describes the 2D area in space and $h$ denotes the vertical height of the room above sea level. We assume the unit for the height is $meter$. Based on Figure \ref{fig:finfraexample}, 
consider an example shown in Figure \ref{fig:indoorexample} that 
a large rectangle bounded by bold lines denotes the vertical view of an office building. 
There is an office room inside denoted by $r_{312}$ with the bounding box plotted by dotted lines. We assume the room height is $9$ (meter). Then, the room is represented by $r_{312}(poly_{312},9.0)(poly_{312} \in D_{\underline{region}}$, we omit the detail description here) and the bounding box of $r_{312}$ is denoted by $bb\_r_{312}(bb_{312},9.0)$($bb_{312}=\{(x,y)|10\leq x\leq 21\wedge 13 \leq y \leq 20\}$). A point $p$ locates in $r_{312}$ which is denoted by $(r_{312},(5,3))$ and $(5,3)$ is the relative value according to $bb\_r_{312}$ (the left lower point of $bb\_r_{312}$ is the origin point). 
But its global coordinate in space can still be transformed by $p$ and $bb\_r_{312}$, 
that is $p'(15,16)$. \\

\begin{figure}[htb]
\psset{unit=1mm}
	\centering
	\begin{pspicture}(0,0)(60,40)
	\psline[arrows=->](5,5)(50,5)
    \psline[arrows=->](5,5)(5,38)
	\put(2,2){\footnotesize{0}}
	\put(9,2){\footnotesize{5}}
%	\put(15,2){\tiny{10}}
%	\put(20,2){\tiny{15}}
%	\put(25,2){\tiny{20}}
%	\put(30,2){\tiny{25}}
%	\put(35,2){\tiny{30}}
	\put(40,2){\footnotesize{35}}
%	\put(45,2){\tiny{40}}
%	\put(50,2){\tiny{45}}
	\put(0,14){\footnotesize{10}}
	\put(0,30){\footnotesize{25}}
%	\put(2,25){\tiny{20}}
%	\put(25,2){\tiny{20}}
	\pspolygon[showpoints=true, linewidth=0.4,dotsize=1.2](10,15)(40,15)(40,30)(10,30)
	\pspolygon[showpoints=true, linewidth=0.3,dotsize=0.8](18,18)(26,22)(23,25)(15,21)
	\pspolygon[linewidth=0.2,linestyle=dotted,dotsize=1](15,18)(26,18)(26,25)(15,25)
	\psdots[dotsize=0.9](20,21)
    \psline[arrows=->,angle=30,linewidth=0.2](20,22)(25,35)
	\put(26,35){\footnotesize{p($r_{312},(5,3)$) \hspace{0.3cm} $\rightarrow p'(15,16)$}}
%	\put(30,35){\tiny{p'(15,21)}}

	\psline[linestyle=dotted,linewidth=0.3](15,5)(15,18)
	\psline[linestyle=dotted,linewidth=0.3](26,5)(26,18)
	\psline[linestyle=dotted,linewidth=0.3](5,25)(15,25)
	\psline[linestyle=dotted,linewidth=0.3](5,18)(15,18)
	\psline[arrows=->,linestyle=dotted](15,18)(30,18)
    \psline[arrows=->,linestyle=dotted](15,18)(15,29)
	\put(0,25){\footnotesize{20}}
	\put(0,18){\footnotesize{13}}
	\put(14,2){\footnotesize{10}}
	\put(25,2){\footnotesize{21}}
	\put(45,25){\footnotesize{$a$ $building$ $in$ $2D$}}
	\put(28,22){\footnotesize{$r_{312}$}}
	\end{pspicture}
	\caption{\label{fig:indoorexample} Relative Position in $Indoor$} 
 \end{figure}

In the $indoor$ environment, the case exists that the height above sea level of a moving object 
changes during its movement. Usually this occurs when the object moves on a staircase,
but it can also happen in an amphitheater or a chamber where one room has several floors with
different altitudes. It results in 2D areas having different height values for one room. 
To be more general, based on Definition \ref{3dregion} we define a 
data type named $\underline{groom}$, given in the following: 

\begin{Statement}
\label{3dgeo}
\ Data Type for General Room 

$D_{\underline{groom}}=2^{Region3d}$

\end{Statement}

Each element in $D_{\underline{groom}}$ is a set of 3D regions. 
Using the staircase as an example, a staircase (an \textit{infrastructure object}) between two 
floors is modeled as a set of 3D regions where 
each represents one footstep of the staircase. For an elevator which can be considered as
a cube in 3D space, we divide it into a set of sub cubes where each corresponds to the
connection between two floors. Thus, the elements of an elevator 
have the same 2D area in space but with different height values. \\


With the data type above, the location in \textit{indoor} environment is represented as follows. 
Let $IO_{indoor}$ denote an \textit{infrastructure object} and 
$u_{indoor}(i,oid,i_{loc_1},i_{loc_2}, m)$ be a temporal unit for \textit{indoor} moving objects.  Applying the definition of \textit{generic temporal units} in Definition \ref{defintiontempunitbus}, the corresponding attributes in $u_{indoor}$ map to: \\

(1) $oid \rightarrow IO_{indoor}(oid$, \textit{GROOM}, $\beta$, $name)(\beta \in D_{\underline{groom}})$;  

(2) $i_{loc_1}\rightarrow (x,y)$; 

(3) $i_{loc_2}\rightarrow (x,y)$; 

(4) $m\rightarrow Indoor$. \\

The coordinate $(x,y)$ denotes the relative position in the 2D area $IO_{indoor}$ projects 
to. The height of a moving object is determined by which \textit{groom} the object is located in. To clarify terms, when we speak of a \textit{groom}, it is a short description for a general room which can be an office room, a chamber, etc., while $\underline{groom}$ denotes the data type representing a \textit{groom}. According to Definition \ref{genrange}, the elements of a trajectory projecting to this infrastructure are specified as: \\

(1) $sub\_traj_i.oid \rightarrow IO_{indoor}(oid$, \textit{GROOM}, $\beta$, $name)(\beta \in D_{\underline{groom}})$;

(2) $sub\_traj_i.l$ records the movement inside $IO_{indoor}.\beta$;

(3) $sub\_traj_i.m \rightarrow Indoor$.  \\

As the position in $indoor$ changes in a 3D space, a 2D point and a line can not be applied for describing the location and trajectory. Two new data types are involved to define them. Let $\underline{point3d}$ be the data type for a 3D point. 

\begin{Statement}
\label{3dpoint}
point3d  

$D_{\underline{point3d}}=\{(x,y,z)|x,y,z,\in D_{\underline{real}}\}$

\end{Statement}

Let $\underline{line3d}$ be the date type defining \textit{indoor} moving objects' trajectories which is a sequence of 3D points. The definition is as follows:

\begin{Statement}
\label{3dline}
line3d 

$D_{\underline{line3d}}=\{<q_1,q_2,...,q_n>| n \in D_{\underline{int}}, n \geq 0$, and $\forall i \in [1,n], q_i \in D_{\underline{point3d}}\}$

\end{Statement} 

The above two are specific data types for the \textit{indoor} environment representing locations and trajectories. Note that the generic data types $\underline{genloc}$ and $\underline{genrange}$ defined in Section \ref{sec:location} cover the \textit{indoor} environment and we can convert from generic data types to specific types. More contents about the coverting are introduced in Section \ref{sec:datatypeconvert}. \\

\textbf{door} \\

Each door occupies a position in space and builds the connectivity between
two $grooms$. As a door is shared by two $grooms$, the location of the door is represented by two 
objects where each denotes the relative position in the $groom$ it belongs to. Note that the absolute 
position in space for them is the same. For each object, let $\underline{doorpos}$ be 
the data type representing it, defined as follows. 

\begin{Statement}
\label{geodoor}
\ Geo-location for Door  

$Doorpos=\{(gr\_id, pos)|gr\_id \in D_{\underline{int}}, pos \in D_{\underline{line}} \}$.

\end{Statement}

$gr\_id$ is a \textit{groom} identifier and $pos$ records the position of the door in 
that groom. In real case, besides the location information the door has another attribute describing whether the connectivity is available, i.e., \textit{open} or \textit{closed}. Sometimes the door is \textit{open} so that people can go through it, but sometimes it is \textit{closed}
(of course you can enter if you have the key. But in this paper we assume people have the 
general authority, in a visitor's context). To model the time-dependent state, we let $\underline{mbool}$ (employed from \cite{FG+00}) describe the temporal attribute, the value of which is represented as 
a sequence of units where each has two values: a temporal interval and a bool value. 
Each unit describes during the time interval the state is \textit{open} (\textit{true}) or \textit{closed} (\textit{false}). We give the data type representing doors in the following. 

\begin{Statement}
\label{door}
\ Data Type for Door  

$D_{\underline{door}}=\{(dp_1, dp_2,tp,genus)|dp_1,dp_2 \in Doorpos,tp \in D_{\underline{mbool}},genus \in \{non$-$lift,lift\}\}$.
\end{Statement}

Each door has $dp_1,dp_2$ representing the relative position in the two $grooms$ it is connecting,
a temporal type $tp$ denoting the time-dependent state (\textit{open} or \textit{closed}) and 
$genus$ describing whether the door is for an elevator or a normal office room door. We distinguish between lift doors and non-lift doors because the time-dependent state for non-lift doors can be known, while the state for lift doors can not. Usually, the state for an office room door or a chamber door has a regular pattern, e.g., it is open from 8:00 am to 6:00 pm and closed at the other time. But for an elevator, the door (entrance/exit) of which at each level can be considered as the connection between different floors, it is unknown at which floor the lift currently stays so the state for an lift door is uncertain. And we set the bool value of each unit in $tp$ as undefined. 

\subsubsection{Navigation}
\label{sec:navigation}
We define a graph model on \textit{grooms} and \textit{doors} for indoor trip planning as follows: 

\begin{Statement}
\label{indoorgraph}
\ A Graph Model for Indoor (\textbf{GMI} for short) 

A \textbf{GMI} is defined as $G_{indoor}(N,E,\sum_{groom},\sum_{door},l_v,l_e,W)$ where: 

\begin{enumerate}
 \item $N$ is a set of nodes;
 \item $E\subseteq V \times V$ is a set of edges;
 \item $l_v: N\rightarrow \sum_{door}$ is a function assigning labels to nodes;
 \item $l_e:E \rightarrow \sum_{groom}$ is a function assigning labels to edges;
 \item each edge is associated with a weight value from W which denotes the shortest 
distance for connecting two endpoints. 
\end{enumerate}
\end{Statement}

Each node represents a door which records the position in two \textit{grooms} it belongs to. Each
edge corresponds to a general room and builds the connection between two nodes 
(doors) without passing through a third door. 
It says that one door is reachable from another via a \textit{groom}. One \textit{groom} can have several edges in the graph if it has more than two doors, because each pair of doors indicates a connection. 
For example, the elevator is a \textit{groom} consisting of several 3D regions, each of which is the cube between two floors. The entrance/exit on each floor is represented as a node and the connection 
between two floors is represented as an edge. 
Each edge is associated with the shortest path (described by a 3D line) between two doors 
inside the \textit{groom} and the weight is the length of the path. The weight may be not the
Euclidean distance as there can be obstacles inside rooms. So, it includes both 
Euclidean distance and obstructed distance \cite{ZPMZ04,YLJ10}. To build the graph, one needs to precompute the distance between two doors inside a groom. Usually, one room does not have too many doors so the cost is not relatively high. \\

Consider a simple example in Figure \ref{fig:fpandcga} which has four rooms $\{gr_1,gr_2,gr_3,gr_4\}$ and four doors 
$\{d_1,d_2,d_3,d_4\}$ where $gr_1,gr_2,gr_3$ can be considered as office rooms and $gr_4$
is a hallway. Objects $d_1,d_2,d_3$ are the corresponding doors for $gr_1,gr_2,gr_3$, and
$d_4$ is the entrance/exit for the hallway. The connectivity graph is depicted in Figure
\ref{fig:fpandcgb}. Each node corresponds to a door which connects two grooms (e.g., $d_1\rightarrow gr_1,gr_4$) except $d_4$. And each edge denotes a room connecting two doors (for simplicity, we omit the shortest path stored in the edge). With our model, it can support searching three types of optimal routes. 
\begin{enumerate}
 \item \textbf{shortest distance}
 \item \textbf{smallest number of rooms}
 \item \textbf{minimum traveling time} \\
\end{enumerate}

\begin{figure}[htb]
%\centering
% \includegraphics[scale=0.1,width=2.8in,height=1.8in]{images/busexample.eps}
\psset{unit=0.7mm}
	\centering
\psset{unit=0.7mm}
	\centering
\subfigure[]{
	\label{fig:fpandcga}
	\begin{pspicture}(0,0)(60,60)
	\psline(5,5)(5,55)(55,55)(55,5)
	\psline(5,5)(20,5)
	\psline(30,5)(55,5)
	%r1
	\rput(12,15){\footnotesize{$gr_1$}}
	\psline(5,35)(20,35)(20,25)(17,22)
	\psline(20,22)(20,5)
	\rput(15,26){\footnotesize{$d_1$}}

	%r2
	\rput(35,50){\footnotesize{$gr_2$}}
	\psline(5,45)(20,45)
	\psline(20,48)(29,45)(55,45)
	\rput(25,50){\footnotesize{$d_2$}}

	%r3
	\rput(45,22){\footnotesize{$gr_3$}}
	\psline(30,5)(30,20)
	\psline(33,20)(30,24)(30,35)(55,35)
	\rput(36,22){\footnotesize{$d_3$}}

	\psline(20,5)(23,5)
	\psline(23,8)(26,5)(30,5)
	\rput(25,10){\footnotesize{$d_4$}}

	%
	\rput(25,27){\footnotesize{$gr_4$}}
	\rput(37,39){\footnotesize{$gr_4$}}
	\end{pspicture}}
\hspace{1cm} 
\subfigure[]{
	\label{fig:fpandcgb}
	\begin{pspicture}(0,0)(80,60)
	\cnodeput(5,25){A}{\footnotesize{$d_1$}}		
	\cnodeput(25,45){B}{\footnotesize{$d_2$}}		
	\cnodeput(45,25){C}{\footnotesize{$d_3$}}		
	\cnodeput(25,10){D}{\footnotesize{$d_4$}}		

	\ncarc[arcangle=40]{A}{B}
	\naput{\footnotesize{$gr_4$}}

	\ncarc[arcangle=40]{B}{C}
	\naput{\footnotesize{$gr_4$}}

	\ncarc[arcangle=40]{C}{D}
	\naput{\footnotesize{$gr_4$}}

	\ncarc[arcangle=-40]{A}{D}
	\nbput{\footnotesize{$gr_4$}}

	\ncarc[arcangle=-10]{A}{C}
	\nbput{\footnotesize{$gr_4$}}

	\ncarc[arcangle=-30]{B}{D}
	\nbput{\footnotesize{$gr_4$}}

	\rput(70,45){\footnotesize{$d_1\rightarrow gr_1,gr_4$}}
	\rput(70,40){\footnotesize{$d_2\rightarrow gr_2,gr_4$}}
	\rput(70,35){\footnotesize{$d_3\rightarrow gr_3,gr_4$}}
	\rput(66,30){\footnotesize{$d_4\rightarrow gr_4$}}
	\end{pspicture}}
	\caption{Floor Plan and Connectivity Graph} 
 \end{figure}

\textbf{shortest distance} \\

As \textbf{GMI} is a graph, we can apply Dijkstra's algorithm to find a route with the 
shortest distance. Note that a preprocessing step is required before applying the normal shortest path algorithm as the input data is a source room and a destination room, which are represented as edges
in the graph. So, it first has to find all doors that the source room has 
which correspond to the nodes in the graph, and then for each node it processes the searching algorithm. For the terminate condition, as each node (door) has two positions where each maps to a room, the process stops when one of the rooms equals to the destination. Because initially there can be several doors available to start searching (if the source room has several doors), 
the result is a set of shortest paths.  
The shortest one is selected among them. To improve the searching procedure, 
the $A^*$ algorithm can be applied where the Euclidean distance between two locations is set 
as the heuristic value. As the position of a door is represented as a 2D line in a groom, 
we take the middle point to compute the distance. Given two points $p$, $q$ in 3D space,
let $p(i)$, $q(i)$ denote the value in dimension $i$ ($i$=$\{1,2,3\}$). 
The distance between $p$ and $q$ is calculated by 

\begin{equation}
\label{bs}
dist(p,q)= (\Sigma_{i=1}^3(p(i)-q(i))^2)^{1/2}
\end{equation}

\textbf{smallest number of rooms} \\

This case is simple. The weight is set as one for each edge because each edge corresponds to a general room. So, the cost is achieved by aggregating the number of edges in the path. \\

\textbf{minimum traveling time} \\

Let $v_{walk}$ be the average speed of a pedestrian walking in \textit{indoor}. As for each edge we already have the shortest path connecting two doors inside a \textit{groom}, the time for passing an edge can be obtained by $v_{walk}$ and the length of the path. This can apply for passing \textit{grooms} like office rooms, chambers, corridors. But it is not available for the time spent on an elevator which contains two parts: \\

(1) time for waiting until the lift door is \textit{open}; 

(2) time on moving in the elevator. \\

The time for the first part is uncertain because it is unknown which floor the elevator locates at
for a given time instant. The second part can be easily acquired if the height between two floors and
the elevator speed is given. We process the two parts as follows. Let $v_{lift}$ be the lift speed which is a constant value and 
$d^{lift}_i$, $d^{lift}_j$ ($\in D_{\underline{door}}$) be two entries for an lift on the floor 
$i$ and $j$. The whole time spent (including waiting) on moving from $d^{lift}_i$ to $d^{lift}_j$ 
depends on the time arriving at $d^{lift}_i$ and the path where the lift moves along to reach 
$d^{lift}_j$. Without loss of generality, we assume $0\leq i<j$. In the best case, the lift 
happens to stay at floor $i$ and it directly moves up to floor $j$. On the contrary, when
it arrives at $d^{lift}_i$ the lift just leaves from floor $i$ and moves up so that it has 
to wait until it goes down to the bottom floor and comes up again. 
Let $h$ be the height between two floors and there is $n$ floors, then the length of 
shortest and longest path between floor $i$ and $j$ is $(j-i)*h$ and $(j-i)*h+2*n*h$, respectively. 
The two values are the lower and upper bounds and the set of all possible values is 
\{$(j-i)*h$, $(j-i)*h+h$, ... ,$(j-i)*h+2*n*h$\}. Each value is associated with a membership probability depending on the elevator schedule. We do not focus on elevator schedule algorithm but modeling the time cost. A possible solution would be the uniform distribution for the probability of each path length. With the length of path and $v_{lift}$, the time cost on an elevator can be obtained. \\

Some indoor graph models \cite{LOS06,JLY209} represent a \textit{groom} as a node and a door 
as an edge. But in our model, we do it the other way round. For the edge which corresponds
to a \textit{groom}, it explicitly defines the shortest path connecting two
doors inside the \textit{groom}. It has some advantages due to the following reasons:  

\begin{enumerate}
 \item To get the shortest path defined on each edge, the cost is small as the calculation is only limited inside a \textit{groom}, and normally a straight line connecting two doors can suffice; 

 \item  It is meaningful to express how people follow the route. 
Explicitly showing the route line for each room is more helpful than just describing which rooms
you should pass through, especially when the room is very large and has complex structures, 
e.g., an airport hall. If the room is simply abstracted as a node, the route inside from one door to another can not be well described. Also the weight value assigned to the door (edge) can not be defined. Additionally, there can be some obstacles inside a groom so that the straight line 
does not work. In this case, some part of the shortest path may be not visual from the position 
where the traveler is currently located. This motivates us to describe the route explicitly for 
better guiding;

 \item We can express three kinds of optimal routes but others are not feasible to express all of them.
 This is because if the node represents a room and the edge denotes a door, it is not able to define the shortest path connecting two rooms on the edge. 
\end{enumerate}


The \textit{Doors Graph} \cite{YLJ10} is similar as ours, but there are still
some differences. First, they do not consider the temporal state for doors: $open$ or $closed$. 
That is the connectivity in \textit{Doors Graph} is constant, while the state of nodes (doors) in our 
graph is time-dependent so that we can model the elevator which is not included by \textit{Doors Graph}. Second, it does not give the data type representing doors and rooms. 
We believe the geometric representation for rooms is important because it has to explicitly 
describe the route from one room to another for navigation. The minimal indoor walking 
distance in \cite{YLJ10} can be applied because the shortest distance between two doors is also computed and assigned to the edge in our model. Third, the concept of doors in our model is general which includes the entrances/exits for elevators and entrances for staircases, 
but \textit{Doors Graph} is only for normal office doors and hallways.
\subsection{Region-based Outdoor}
\label{regionbased}

For this environment, we employ the method in \cite{MR05,MRS05} that the space is 
partitioned into a set of disjoint zones where each is represented by a region. The 
location of a moving object is represented by \textit{referencing} to these zones. 
The partition algorithm is beyond the scope of present paper. 
\cite{MR05,MRS05} only represents the location in an approximate way, that is it is known
which zone the object is located in, but the precise location inside is not represented. Here,
we give the representation for precise location. It is known which zone the moving object is 
located in as well as the relative position inside. 
As there are already infrastructures for public transportation modes 
(Section \ref{publictn}) and modes like $Car$ and $Taxi$ (introduced in the next subsection)
where all of them are for $outdoor$ movements, the infrastructure here only serves for the mode 
$Walk$. Let $S$ be the overall outdoor space for walking and it is partitioned into a set of 
zones $\{S_1,S_2,...,S_n\}(n \in D_{\underline{int}})$ where \\

(1) $\forall i\in [1,n],S_i\neq\oslash$; 

(2) $\forall i,j \in[1,n], i \neq j \Rightarrow S_i\cap S_j=\oslash$; 

(3) $\bigcup S_i=S$.\\

\begin{figure}[htb]
%\centering
% \includegraphics[scale=0.1,width=2.8in,height=1.8in]{images/busexample.eps}
\psset{unit=0.7mm}
	\centering
	\begin{pspicture}(0,0)(100,80)
	%block_1
	\pspolygon[fillstyle=crosshatch,hatchwidth=0.2](5,45)(35,45)(35,50)(5,50)
%	\rput(20,47){\footnotesize{$building_1$}}

	%block_2
	\pspolygon[fillstyle=crosshatch,hatchwidth=0.2](5,19)(35,19)(35,35)(5,35)

	%block_3 	
	\pspolygon[fillstyle=crosshatch,hatchwidth=0.2](5,7)(35,7)(35,13)(5,13)

	%block_4 
	\pspolygon[fillstyle=crosshatch,hatchwidth=0.2](45,5)(65,5)(65,15)(45,13)

	%block_5 
	\pspolygon[fillstyle=crosshatch,hatchwidth=0.2](45,24)(65,22)(70,33)(70,45)(50,40)
%	\rput(58,30){\footnotesize{$building_4$}}

	%block_6 
	\pspolygon[fillstyle=crosshatch,hatchwidth=0.2](75,53)(50,48)(52,55)(77,60)
	
	\psline(5,60)(35,60)(52,65)(77,70)
	\psline(5,0)(35,0)(45,1)(70,1)

	\psline[linestyle=dashed, linewidth=0.3](35,50)(52,55)
	\psline[linestyle=dashed, linewidth=0.3](35,7)(45,5)
	\psline[linestyle=dashed, linewidth=0.3](35,45)(35,35)
	\psline[linestyle=dashed, linewidth=0.3](35,19)(35,13)
	\psline[linestyle=dashed, linewidth=0.3](45,13)(45,24)
	\psline[linestyle=dashed, linewidth=0.3](50,40)(50,48)
	\psline[linestyle=dashed, linewidth=0.3](5,0)(5,7)
	\psline[linestyle=dashed, linewidth=0.3](5,13)(5,19)
	\psline[linestyle=dashed, linewidth=0.3](5,35)(5,45)
	\psline[linestyle=dashed, linewidth=0.3](5,50)(5,60)
	\psline[linestyle=dashed, linewidth=0.3](70,1)(65,5)
	\psline[linestyle=dashed, linewidth=0.3](65,15)(65,22)
	\psline[linestyle=dashed, linewidth=0.3](70,45)(75,53)
	\psline[linestyle=dashed, linewidth=0.3](77,60)(77,70)

	\rput(32,55){\footnotesize{$pl_1$}}
	\rput(22,40){\footnotesize{$pl_2$}}
	\rput(25,16){\footnotesize{$pl_3$}}
	\rput(30,3){\footnotesize{$pl_4$}}
	\rput(42,30){\footnotesize{$pl_5$}}
	\rput(60,45){\footnotesize{$pl_6$}}
	\rput(56,18){\footnotesize{$pl_7$}}
	\psline[arrows=->](12,52)(38,52)(38,18)(50,18)
	\put(12,54){\tiny{$P_i$}}
	\end{pspicture}
	\caption{\label{fig:outdoorexample} Outdoor Space Partition} 
 \end{figure}

Each zone represents an area in space for people walking, i.e.., the pavement or footpath. 
Figure \ref{fig:outdoorexample} shows an example of a partition on
the city map where the polygons drawn by crosshatching represent the buildings and the 
others represent the area that people can walk in. We only consider the outdoor
movement, thus the space for walking (ignoring the building areas) consists of 
seven disjoint polygons $\{pl_1,...,pl_7\}$. 
The shape of a polygon can be regular, e.g., $pl_2,pl_3$, or irregular, $pl_4,pl_6$. \\

The location is represented in two steps: first, we map the location to a polygon; second, the relative position in the 
polygon is identified. We use $IO_{rbo}$ to represent an \textit{infrastructure object} in 
$I_{rbo}$ and let $u_{rbo}(i,oid,i_{loc_1},i_{loc_2},m)$ denote a temporal unit for moving objects with the transportation mode $Walk$ where  \\

(1) $oid \rightarrow IO_{rbo}(oid$, \textit{REGION}, $\beta$, $name) (\beta \in D_{\underline{region}})$; 

(2) $i_{loc_1} \rightarrow (x,y)$;

(3) $i_{loc_2} \rightarrow (x,y)$;

(4) $m \rightarrow Walk$. \\

During time interval $i$, the position between $i_{loc_1}$ and $i_{loc_2}$ is obtained by \textit{linear interpolation}. Using Definition \ref{lowresolutionunit}, we can also 
represent the location in a low resolution way. That is, the location only goes to the level of a
polygon identification, while the accurate location inside is ignored 
($i_{loc_1}\rightarrow (\bot,\bot), i_{loc_2}\rightarrow (\bot,\bot)$). 
Consider an example movement of a pedestrian in Figure 
\ref{fig:outdoorexample}, denoted by $traj_1$. With the low resolution, we can represent 
it by $P_i=<(pl_1,Walk),(pl_5,Walk),(pl_7,Walk)>$ 
(for brevity, we ignore the time interval $i$ and location description $i_{loc_1},i_{loc_2}$). 
This low resolution representation is used for mobility pattern queries \cite{MR05}. 


\subsection{Road Network}
\label{sec:roadnetwork}

For this part, we directly transfer the method in \cite{GA2006} where the street, 
highway and road is described by a $\underline{line}$ with an unique identifier. 
As objects move on these $\underline{line}$ objects, the location is represented 
by a two-tuple $(rid, pos) (rid \in D_{\underline{int}}, pos \in D_{\underline{real}})$ 
where $rid$ is the route id and $pos$ denotes the relative position in that route.  
More details can be found by referring to that paper. Applying the framework of 
\textit{generic temporal units}, let $IO_{rn}$ denote a street or highway and $u_{rn}(i,oid,i_{loc_1},i_{loc_2},m)$ be a temporal unit for moving objects in road network. Specifically, \\

(1) $oid \rightarrow IO_{rn}(oid$, \textit{LINE}, $\beta$, $name)(\beta \in D_{\underline{line}})$; 

(2) $i_{loc_1} \rightarrow (pos,\perp)$;

(2) $i_{loc_2} \rightarrow (pos,\perp)$;

(4) $m\rightarrow \{Car,Taxi, Bicycle\}$.  \\

To identify a position along a $line$, only one dimensional data is needed. 
Thus, for the second attribute $\delta_2$ in $i_{loc_1}$ and $i_{loc_2}$, 
we let it be undefined. The positions between $i_{loc_1}.pos$ and $i_{loc_2}.pos$ are achieved by \textit{linear interpolation}. 

\subsection{Free Space}
\label{sec:freespace}
In this infrastructure, there is no element inside. That is, the location is directly
represented by the precise position in space (geographical coordinates) where no 
\textit{infrastructure object} is \textit{referenced} to. As in free space there is no limitation on transportation modes, any mode is available. Using the location representation presented in \cite{FG+00,GBE+00}, let $u_{fs}(i,oid,i_{loc_1},i_{loc_2},m)$ be a temporal unit for moving objects in free space and the location between $i_{loc_1}$ and $i_{loc_2}$ is obtained by \textit{linear interpolation}. Specifically, \\

(1) $oid \rightarrow \bot$; 

(2) $i_{loc_1} \rightarrow (x,y)$;

(3) $i_{loc_2} \rightarrow (x,y)$;

(4) $m\rightarrow m' (m' \in D_{\underline{tm}})$.

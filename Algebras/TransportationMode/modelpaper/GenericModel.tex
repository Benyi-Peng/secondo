\section{Generic Model}
\label{sec:genericmodel}
In this section, we present the generic location representation for moving objects in various environments. To define the data types in our model, we employ some basic data types in \cite{FG+00}
which are introduced in Section \ref{sec:preliminary}. Section \ref{sec:datamodel} introduces the transportation modes that we consider, defines a finite set of \textit{intrastructures} and gives the generic location representation. The framework for generic moving objects is presented in Section \ref{sec:trajectoryrepresentation} and the multiresolution location representation is discussed in Section \ref{sec:multi-scaleresolution}.

\subsection{Preliminary}
\label{sec:preliminary}
We give the carrier set of basic types that we use for the definitions in the following sections 
\footnote{Using the algebraic terminology 
that for a data type $\alpha$, its domain or carrier set is denoted as $D_{\alpha}$.}. 

\begin{Statement}
\label{basictype}
\ Base Types 

$D_{\underline{int}} = Z \cup \{\perp\}$

$D_{\underline{real}} = \Re \cup \{\perp\}$

$D_{\underline{bool}} = \{FALSE,TRUE\} \cup \{\perp\}$

\end{Statement}

Each domain is based on the usual interpretation with an extension by the undefined value $\{\perp\}$. 
We give three data types representing time: \textit{time instant}, \textit{time interval} and \textit{time range}, defined as follows: 

\begin{Statement}
\label{timetype}
\ Time Types  

time instant: $D_{\underline{instant}} = \Re \cup \{\perp\}$

time interval: $D_{\underline{interval}}$ = $\{(s,e,lc,rc)|s,e \in D_{\underline{instant}}, lc, rc \in D_{\underline{bool}}, s\leq e, (s=e) \Rightarrow (lc=rc=true)\}$

time range: $D_{\underline{periods}}=\{V \subseteq D_{\underline{interval}}|$
$(u,v \in D_{\underline{interval}} \wedge u\neq v)\Rightarrow$ disjoint(u,v) $\wedge$ $\neg adjacent(u,v) \}$
\end{Statement}

We use $\underline{instant}$ to denote the data type for time instant, which is based 
on the $\underline{real}$ type. The value of a time interval is represented by two end 
instants and two flags $lc$ and $rc$ indicating whether at the end instant it is closed or open 
where the start instant $s$ should be no later than the end 
instant $e$. It defines a set of time instants. 
Type $\underline{periods}$ represents a set of disjoint and non-connected time intervals. We also
give the $\underline{intime}$ type constructor which yields types whose values consist of a time 
instant and a value. Let $\alpha$ denote an abstract type (excluding time type) which can be $\underline{int}$, $\underline{real}$, etc. 

\begin{Statement}
\label{intime}

$D_{\underline{intime}} = D_{\underline{instant}} \times D_{\alpha}$

\end{Statement}

In addition, we also employ three spatial data types, $\underline{point}$, $\underline{line}$ and $\underline{region}$, and two temporal types $\underline{mbool}$ (moving bool) and 
$\underline{mpoint}$ (moving point). Their 
corresponding domains are denoted by $D_{\underline{point}}$, $D_{\underline{line}}$, $D_{\underline{region}}$, $D_{\underline{mbool}}$ and $D_{\underline{mpoint}}$ 
(for the technical definition we refer readers to \cite{FG+00,GBE+00}). 

\subsection{Data Model}
\label{sec:datamodel}

Using the \textit{object-based} method, we let the $space$ where an object moves be covered by
a set of \textit{infrastructures} (introduced in detail in Section \ref{sec:infrastructure}). 
Then, the location of a moving object is represented by mapping it to these \textit{infrastructures}. 
Each \textit{infrastructure} covers some places and is also related to a kind 
of transportation mode. For example, if the \textit{infrastructure} is road network, the mode 
can be $Car$ \footnote{we use Italics font to denote a kind of transportation mode.}, 
$Taxi$, etc. If the \textit{infrastructure} is the area such as the pavement or footpath, the mode is $Walk$. Table \ref{tab:infrastructure} lists the classification of movements on available places 
as well as the corresponding transportation modes. For a special case that the 
movement has no constraint in the environment (i.e., free space), we let any 
mode be available. 

\begin{table}[ht]
 \begin{center}
  \begin{tabular}{c|c}
	\hline
	\textbf{Places}& \textbf{Transportation Modes}\\ 
	\hline
	bus routes& $Bus$\\
	\hline
    railways& $Train$, $Tube$\\
    \hline
	buildings& $Indoor$\\
	\hline
	pavements, footpaths& $Walk$\\	
	\hline
	roads, highways, streets& $Car$, $Bicycle$, $Taxi$\\
	\hline
	free space & any mode\\
	\hline
  \end{tabular}
 \end{center}
 \caption{\label{tab:infrastructure}Classification of Movements}
\end{table}

Let $\underline{tm}$ denote the data type for transportation modes, whose carrier set 
is defined in the following. 
 
\begin{Statement}
\label{tm}
\ Transportation Mode 

$D_{\underline{tm}} =\{Car, Bus, Train, Walk, Indoor, Tube, Taxi, Bicycle\}$

\end{Statement}

It can be considered as an enum type. As people also walk in the $indoor$ space, in the
following when we say mode $Walk$ it means outdoor environment by default, which is to 
distinguish the modes betweeen $Walk$ and $Indoor$. Here, we only consider transportation modes for 
objects moving on the underground. Of course, there are still two kinds of transportation modes: $Ship$ and $Airplane$. Compared with the modes listed in Table \ref{tab:infrastructure}, normally people do not frequently change to these two modes in daily life. So, we don't consider them in this paper but they can be easily integrated because both of them can be considered as public transportation vehicles. 

\subsubsection{Infrastructures and Infrastructure Objects}
\label{sec:infrastructure}
Each \textit{infrastructures} describes a kind of environment for objects moving and it is composed of
a set of elements called \textit{infrastructure objects}. 
Each object is represented by a value where the type of the value
is according to the \textit{infrastructure} characteristic. 
For instance, in road network (an \textit{infrastructure}) a $line$ value is used to describe the 
geometrical property of a road or a highway, and in free space (also considered as an \textit{infrastructure}) a $polygon$ or a $region$ 
(in the following, we use terms $polygon$ and $region$ 
interchangeably) is used to identify the pavement area for people walking. 
Besides, the \textit{infrastructure} can also be an environment where the elements are moving objects, 
such as buses and trains in the public transportation network. The location of moving 
objects (specifically, humans and travelers) can be represented by \textit{referencing} to these \textit{infrastructure objects}. 
For example, streets and highways are treated as \textit{infrastructure objects} which constitute
the \textit{infrastructure} road network. Then, the location of a driver in a car moving on a highway can be represented by a two-tuple $(rid,pos)$ where $rid$ corresponds to the highway and $pos$ denotes the relative position along that highway \cite{GA2006}. In mobility pattern queries \cite{MR05}, 
they consider objects moving in a free space and partition the space into a set of disjoint zones (\textit{infrastructure objects}) each of which is represented by a $region$ and for each zone 
a label is assigned. Thus, the location is represented by mapping it to these zones. 
According to the movement classification in Table \ref{tab:infrastructure}, a taxonomy of 
\textit{infrastructures} considered in this paper is listed as below: \\


(1) $I_{ptn}:public$ $transportation$ $network$ ($ptn$ for short); \\

(2) $I_{indoor}:Indoor$; \\

(3) $I_{rbo}:Region$-$based$ $Outdoor$ ($rbo$ for short); \\

(4) $I_{rn}:Road$ $Network$ ($rn$ for short); \\

(5) $I_{fs}:free$ $space$ ($fs$ for short). \\

We believe they cover all underground cases. For $I_{ptn}$, \textit{infrastructure} 
\textit{objects} are buses, trains and tubes. For $indoor$, such as rooms, staircases, hallways are the elements, and for $I_{rbo}$ the partitioned 
zones compose the \textit{infrastructure}. The road network $I_{rn}$ consists of streets, highways, 
etc. Free space is an empty \textit{infrastructure}, i.e., no \textit{infrastructure objects}. The detailed representation for each \textit{infrastructure} is presented in Section \ref{sec:RfI}. 
As stated above, for each \textit{infrastructure}, its elements are associated with a value of a data type representing the background geographic information. To be more general, let $\alpha$ be a general data type where the specific value we consider for each infrastructure is listed as below: \\

(1) $I_{ptn}$: $\alpha \rightarrow \underline{mpptn}$ (introduced in Section \ref{publictn}) \\

(2) $I_{indoor}$: $\alpha \rightarrow \underline{groom}$ (introduced in Section \ref{sec:indoor}) \\

(3) $I_{rbo}$: $\alpha \rightarrow \underline{region}$ \\

(4) $I_{rn}$: $\alpha \rightarrow \underline{line}$ \\

First, we use the type $\underline{mpptn}$ representing \underline{m}oving \underline{p}oints in \underline{p}ublic \underline{t}ransportation \underline{n}etwork, 
such as buses, trains and underground trains. In this case, 
the movement of people can be represented by \textit{referencing} to the bus/train they take. 
Second, for $indoor$ environment, basically it consists of 3D objects, such as rooms, 
corridors and staircases where the movement can be both horizontal and vertical. 
We use the type $\underline{groom}$ to describe 3D \textit{infrastructure objects}.
Next, a polygon is used to describe the area covered by the pavement and footpath, 
which is for the mode $Walk$. Last, a $line$ is the abstract description for the street, 
road, or highway. It is for $Car$, $Bicycle$, and $Taxi$. As $I_{fs}$ is an empty set, no data type
is needed.  \\

To be clear with data type notations, we define a symbol for each data type. Let $IOType$ be the set of all data types for infrastructure objects and $IOSymbol$ be a set where its elements are the symbols for infrastructure objects data types, defined as below.  \\

\begin{Statement}
\label{infraobjectsymbol}
\ Infrastructure Objects Data Types and Symbol Data Types 

IOType = \{\underline{mpptn}, \underline{groom}, \underline{region}, \underline{line}, $\bot$ \}

IOSymbol = \{MPPTN, GROOM, REGION, LINE, FREESPACE\}

\end{Statement}

Let $\mu:IOSymbol \rightarrow IOType$ be a mapping from symbols to the full data types, e.g., 
$\mu(MPPTN)\rightarrow \underline{mpptn}$. In \textit{free space}, there is no infrastructure object so that we let the data type be $\bot$. Then, we give the definition for general  \textit{infrastructure object} which consists of four attributes. \\

\begin{Statement}
\label{infraobject}
\ General Infrastructure Object 

Let $IO(oid,s,\beta, name)$ denote an 
infrastructure object where $oid (\in D_{\underline{int}})$ is the unique object identifier, $s(\in IOSymbol)$ is a symbol for a data type, $\beta (\in D_{\mu(s))}$ is the value of a certain infrastructure object data type, and a string value $name$ is to describe the name of the object.
\end{Statement}


Let $I=\{I_{ptn},I_{indoor},I_{rbo},I_{rn},I_{fs}\}$ be the set 
of all \textit{infrastructures}, and $Dom(I_i)(I_i \in I)$ be the set of values 
(\textit{infrastructure objects}) for $I_i$. Then, we define the $space$ where an 
object moves as follows: 

\begin{Statement}
\label{space}

A $space$ is defined as a set of infrastructure objects from all infrastructures, 
denoted by
\[
 Space=\bigcup_{I_i \in I}Dom(I_i)
\]
\end{Statement}

Let $\underline{space}$ be the data type representing the space, its domain is: \\

$D_{\underline{space}}= Space$ \\

We let the space be comprised of a set of \textit{infrastructure objects} which can be 
\textit{static} or \textit{dynamic}. 
The \textit{static} components correspond to spatial objects, e.g., $regions$ or $lines$ and the 
\textit{dynamic} part is for public transportation vehicles, such as buses and trains. Based on
the \textit{full} space (covered by \textit{infrastructure objects}), in the next subsection 
we propose the general method to represent locations for moving objects. 

\subsubsection{Location}
\label{sec:location}

Using the \textit{reference} method, there is a simple way for representing the location. 
That is, the location is represented by a function from time to an 
\textit{infrastructure object}. But there is a significant drawback that the location is 
imprecisely described, i.e., it is only 
at the level of an object, e.g., a $region$. It is also necessary to know the exact and precise location in that object. Assuming the \textit{referenced} objects now are the partitioned zones, 
queries like ``\textit{find all trips moving from zone $A$ to zone $B$ passing a 
location $p$ in $A$}" can not be answered because it is unknown how the object moves 
inside $A$. To solve this problem, we propose that 
the \textit{infrastructure object} should be combined with a second value 
representing the precise location within that \textit{referenced object}. 
For the second value, we simply use a two-tuple $i_{loc}(x,y)$ to denote the position.
The value $i_{loc}$ is according to the \textit{referenced (infrastructure) object} instead of the coordinate in the whole space. \\

Consider a simple example in Figure \ref{fig:finfraexample} where there is 
a polygon denoted by $pl=\{(x,y)|5\leq x\leq35 \wedge 10 \leq y \leq 25\}$ covering a 
certain area (e.g., a plaza) in space and a point $p$ locating inside $pl$. 
Using the above method, we represent $p$ by $(pl,(15,10))$ where 
$(15,10)$ denotes the local position in $pl$ (we let the left lower point 
of the bounding box built on $pl$ be the origin point). Alternatively, if it uses 
the global coordinate value in space, the point is denoted by $p'(20,20)$. 
In Section \ref{sec:RfI}, we present in detail how the location $i_{loc}$ 
is represented for each \textit{infrastructure}. Because for different infrastructures, the values for describing \textit{infrastructure objects} have different data types, e.g., $\underline{line}$, 
$\underline{region}$. \\

\begin{figure}[htb]
\psset{unit=1mm}
	\centering
	\begin{pspicture}(0,0)(60,40)
	\psline[arrows=->](5,5)(50,5)
    \psline[arrows=->](5,5)(5,38)
	\put(2,2){\footnotesize{0}}
	\put(10,2){\footnotesize{5}}
%	\put(15,2){\tiny{10}}
%	\put(20,2){\tiny{15}}
%	\put(25,2){\tiny{20}}
%	\put(30,2){\tiny{25}}
%	\put(35,2){\tiny{30}}
	\put(40,2){\footnotesize{35}}
%	\put(45,2){\tiny{40}}
%	\put(50,2){\tiny{45}}
	\put(0,15){\footnotesize{10}}
	\put(0,30){\footnotesize{25}}
	\pspolygon[showpoints=true, linewidth=0.3](10,15)(40,15)(40,30)(10,30)
	\psdots[dotsize=0.8](25,25)
	\put(24,27){\footnotesize{p($pl$, (15,10))}}
	\put(27,23){\footnotesize{p'(20,20)}}
	\psline[linestyle=dotted,linewidth=0.3](25,5)(25,25)
	\psline[linestyle=dotted,linewidth=0.2](5,25)(25,25)
	\put(0,25){\footnotesize{20}}
	\put(25,2){\footnotesize{20}}
	\put(45,25){\footnotesize{$pl$}}
	\end{pspicture}
	\caption{\label{fig:finfraexample} location description in an \textit{infrastructure object}} 
 \end{figure}

To sum up, we represent the location by two parts where the first is an 
object identifier corresponding to an \textit{infrastructure object} and the second stands for  
the relative position in that object. More formally, it is defined as follows. 


\begin{Statement}
\label{genericloca}
\ Generic Location 

$Loc=\{(\delta_1,\delta_2)|\delta_1,\delta_2 \in D_{\underline{real}}\}$

$D_{\underline{genloc}}=\{(oid,i_{loc})|oid \in D_{\underline{int}},
i_{loc} \in Loc\}$ 
\end{Statement}

We call it \textit{Generic Location} as it can represent locations of moving objects in all environments listed in Table \ref{tab:infrastructure}.  
Each element in $Loc$ has two attributes denoted by $\delta_1$, $\delta_2$ for  
the position description. We do not use symbol $x$ and $y$ because the semantic has some
other meaning (introduced in Section \ref{sec:RfI}) other than only coordinate value. Given 
$i_{loc1},i_{loc2} \in Loc$, we define \\

$i_{loc1}=i_{loc2} \Leftrightarrow i_{loc1}.\delta_1=i_{loc2}.\delta_1 
\wedge i_{loc1}.\delta_2=i_{loc2}.\delta_2$.  \\
 

For the elements in $D_{\underline{genloc}}$, each consists of two attributes where $oid$ is the 
infrastructure object identifier and $i_{loc}$ describes the position in the object.
The representation supports different coordinate reference systems where $global$ and 
$local$ representation can be translated to each other (seeing the example in Figure \ref{fig:finfraexample}). Based on Definition \ref{genericloca}, now we give the representation for 
moving objects location. 

\begin{Statement}
\label{abstractm}
Let $f$ be the location function projecting from time to a location. 
The definition is as follows: \\

\hspace{2cm} $f: D_{\underline{instant}} \rightarrow D_{\underline{genloc}}$
\end{Statement}

The continuous changing location data is represented by a
function $f$ where $D_{\underline{instant}}$ (domain) denotes the time instant and 
$D_{\underline{genloc}}$ (range) represents the set of elements for locations. 
Different from the method in \cite{FG+00,GBE+00} which maps the time to a space 
domain which is for free space, here the elements in $D_{\underline{genloc}}$ can represent locations in multiple environments. Each element includes the space data that is achieved by a two level representation (imprecise and precise). It is richer than \cite{GBE+00} in two respects: 
\begin{itemize}
 \item It is a generic data model that can be applied for various moving environments, 
e.g., \textit{spatial network}, \textit{indoor}. The location representation is not limited in a specific 
surrounding but can cover multiple cases (introduce in detail in Section \ref{sec:RfI}). 
Both \textit{global} and \textit{local} coordinate reference systems are supported. 
At the same time, the topology relations like ``\textit{contain}" and ``\textit{inside}" can be 
derived by $oid$. The model covers this aspect explicitly while involving geometric operations results in costly computation. Besides, the existing queries are preserved because the precise location 
is represented.
 
 \item The location is represented in a multiresolution way. The method of using the object identifier
is an approximate location description, while the precise data is also managed by $i_{loc}$. This 
makes a flexible way to describe objects' movement, while the resolution
depends on the specific application requirement. 
\end{itemize}

Next, we define a data type called $\underline{genrange}$ representing moving objects' trajectories which are curves in space (rationally parameterized by time). The definition is given below. 

\begin{Statement}
\label{genrange}
\ Range Location

$Subrange$ = \{$(oid,l,m)|oid\in D_{\underline{int}},
l \in D_{\underline{line}}, m \in D_{\underline{tm}}$\}

$D_{\underline{genrange}}$ = \{$V \subset Subrange$\}

\end{Statement}

Each $\underline{genrange}$ object is a set of elements each of which is denoted by $sub\_traj_i$ $(\in Subrange)$. $sub\_traj_i$ is composed of three attributes: $oid$ is the infrastructure object identifier, $l$ denotes all locations according to the object and $m$ describes the transportation mode. The trajectory is represented based on the referenced \textit{intrastructure objects} where $sub\_traj_i$ records the movement according to each object as well as the transportation mode. Although the trajectories of indoor moving objects are 3D data , 
$\underline{genrange}$ can still represent them because the projection is to a 2D space with the object identifier (introduced in detail in Section \ref{sec:indoor}).
For a moving object, its trajectory can cover several \textit{infrastructure objects} where they can belong to the same kind of infrastructure or different kinds of infrastructures. We explain how $\underline{genrange}$ applies for all proposed infrastructures. \\
\begin{itemize}
 \item $I_{ptn}$: introduced in Section \ref{publictn}. 

 \item $I_{indoor}$: introduced in Section \ref{sec:indoor}. 

 \item $I_{rbo}$: 

  (i) $sub\_traj_i.oid \rightarrow IO_{rbo}(oid$, \textit{REGION}, $\beta$, $name)(\beta \in D_{\underline{region}})$ 

  (ii) $sub\_traj_i.l$ records the polyline inside $IO_{rbo}.\beta$

  (iii) $sub\_traj_i.m \rightarrow Walk$ 

 \item $I_{rn}$:  

 (i) $sub\_traj_i.oid \rightarrow IO_{rn}(oid$, \textit{LINE}, $\beta$, $name)(\beta \in D_{\underline{line}})$

 (ii) $sub\_traj_i.l$ stores the movement along $IO_{rn}.\beta$ 

 (iii) $sub\_traj_i.m \rightarrow \{Car$, $Taxi$, $Bicycle\}$ 

 \item $I_{fs}$:

  (i) $sub\_traj_i.oid \rightarrow \bot$

  (ii) $sub\_traj_i.l$ records the movement in space

  (iii) $sub\_traj_i.m \rightarrow m' (m'\in D_{\underline{tm}})$
\end{itemize}

\subsection{Moving Objects Representation}
\label{sec:trajectoryrepresentation}
In this paper, we focus on the complete history movement of 
moving objects which can also be called \textit{trajectory}, instead of current and future position. 
To represent the continuous changing location data, a standard way to model it is by the
\textit{regression} method that first segments the data into pieces or regular intervals within 
which it exhibits a well-defined trend, and then chooses the basis and mathematical functions 
most appropriate to fit the data in each piece \cite{FG+00,GRS00,TM08}. 
Using the method of \textit{sliced representation} \cite{FG+00,LFGS03}, we represent 
a moving object as a set of so-called \textit{temporal units (slices)}. 
Each $unit$ defines a time interval as well as the movement during it. First, we give
a general definition for temporal units that applies to all units discussed in this paper.
Let $R$ be a set. The definition of \textit{abstract temporal units} is given as follows: 

\begin{Statement}
\label{generaltemporalunit}
\ Abstract Temporal Units 

$Atu$ = $D_{\underline{interval}}$ $\times$ R
\end{Statement} 

The values of abstract temporal units 
have two components where the first defines a time interval and the 
second identifies an element from a set. Here a pair or unit $(i,r) \in Atu$ 
($i \in D_{\underline{interval}}, r \in R$) says that during the time interval $i$, the movement of 
the unit is identified by $r$. The specific representation for $r$ is introduced 
progressively in the following. Applying the above framework, we define the so-called 
\textit{generic temporal units} for moving objects as below: 

\begin{Statement}
\label{generictemporalunit}
\ Generic Temporal Units

$Gentu=\{(i,oid,i_{loc_1},i_{loc_2}, m)|i \in D_{\underline{interval}},
oid \in D_{\underline{int}}, i_{loc_1},i_{loc_2} \in Loc, m \in D_{\underline{tm}}\}$
\end{Statement}

The value of each element has five components where $i$ defines a time interval, 
$oid$ maps to an infrastructure object, $i_{loc_1}$, $i_{loc_2}$ identify two positions in the object 
and $m$ describes the transportation mode. Considering Definition \ref{generaltemporalunit}, 
for an abstract unit $(i,r)$, here $r$ applies to the part $(oid,i_{loc_1},i_{loc_2},m)$. 
Each unit references to an infrastructure object with a certain transportation mode during the 
time interval. It says that at time instant $i.s$ the object is located at $i_{loc_1}$ in the \textit{infrastructure object} $oid$ and at time instant $i.e$ it moves to $i_{loc_2}$ in that \textit{infrastructure object} with the transportation mode $m$. The position during time
interval $[i.s,i.e]$ is achieved by a discrete function associated with the object $oid$. 
Notice that the mathematical functions manipulating on the data are diverse for different infrastructure objects (introduced in Section \ref{sec:RfI}). Given two units $u_1, u_2 \in Gentu$ that cover the same time interval ($u_1.i=u_2.i$), some relationships can be exploited between them: 

\begin{enumerate}
 \item $u_1.m\neq u_2.m$
  
  It denotes two units with different transportation modes. For example, $u_1.m$ = $Car$ and 
$u_2.m$ = $Walk$. Queries on transportation modes extract data from this attribute. \\

 \item $u_1.m=u_2.m \wedge u_1.oid\neq u_2.oid$ 

The transportation modes are the same, but the \textit{referenced} infrastructure objects are different. Let $u_1.m$ = $Bus$, $u_2.m$ = $Bus$, then $u_1.oid$  and $u_2.oid$ denote different
buses. \\

 \item $u_1.m=u_2.m \wedge u_1.oid=u_2.oid$ 

In this case, both transportation modes and infrastructure objects are the same. But the
precise location in space can be different. For example, two pedestrians walk on the
same pavement or two clerks walk inside the same office room. They can have different
locations in space which are distinguished by $i_{loc_1}$ and $i_{loc_2}$. \\
\end{enumerate}

Given two generic temporal units $u_1,u_2 \in Gentu$ (assuming $u_1.i<u_2.i$),  
let $sub\_traj_1,sub\_traj_2 (\in Subrange)$ denote their projection movements in space and we define \\

$u_1$ and $u_2$ is \textit{mergeable} $\Leftrightarrow$ \\

(i) $u_1.i$ is \textit{adjacent} to $u_2.i$ $\wedge$ $u_1.oid$ = $u_2.oid$ $\wedge$ $u_1.m$ = $u_2.m$ 

(ii) $sub\_traj_1.l$ is \textit{adjacent} and \textit{co-linear} with $sub\_traj_2.l$ \\

Based on the generic temporal units, we give the definition of generic moving objects.

\begin{Statement}
\label{generictrajectory}
\ Generic Moving Objects

$D_{\underline{genmo}}$ = \{$<u_1,u_2,...,u_n>|n \geq 0,n \in int$, and

(i) $\forall i \in [1,n], u_i \in Gentu$ 

(ii) $\forall i,j \in [1,n], i\neq j \Rightarrow u_i.i \cap u_j.i = \oslash$ 
$\wedge$ $u_i,u_j$ is not mergeable \}
\end{Statement}

Each moving object consists of a sequence of temporal units where each unit corresponds to an infrastructure object, the movement in that infrastructure 
object and a kind of transportation mode. 
We only consider the case that the geometry property of a moving 
object is abstracted as a $moving$ $point$ (i.e., focus on the position but ignore the 
extent variation of the object), while $moving$ $line$ and $moving$ $region$ is out scope 
of this paper. Within this framework, we can describe objects' movement in various 
environments. 

\subsection{Multi-Scale Representation}
\label{sec:multi-scaleresolution}
In our model, we represent the location data in a multi-scale way: the imprecise level
by only referencing to an \textit{infrastructure object} and the precise level by both the 
\textit{infrastructure object} and the relative position inside it. 
Each \textit{infrastructure object} is associated with a value representing the space 
it is covering, e.g., a $line$. By \textit{referencing} to it, we know that the object's position is
bounded or inside that area, which is an approximate value. If both levels are achieved, 
the accurate location of moving objects is known. But for some applications 
\cite{MRS05,MR05}, the full resolution for location data may be not necessary so that it 
is much simpler and more efficient to work with the approximate location. 
Thus, we can only represent the location by \textit{referencing} to an infrastructure object, 
while the precise movement inside is ignored. Applying the framework of \textit{generic temporal units} in Definition \ref{generictemporalunit}, we give the definition of so-called \textit{low resolution temporal units} as follows. 

\begin{Statement}
\label{lowresolutionunit}
\ Low Resolution Temporal Units

$Lowgentu=\{(i,oid,i_{loc_1},i_{loc_2}, m)|$

(i) $i \in D_{\underline{interval}}$;

(ii) $oid \in D_{\underline{int}}$;
 
(iii) $i_{loc_1},i_{loc_2} \in Loc, i_{loc_1}=(\bot,\bot) \wedge i_{loc_2}=(\bot,\bot)$;

(iv) $m \in D_{\underline{tm}}\}$

\end{Statement}

It's a subset of the domain on \textit{generic temporal units} where the value in $i_{loc_1}$,
$i_{loc_2}$ is set as undefined, while the framework of \textit{generic moving objects} still 
applies. In Section \ref{regionbased}, we will show an example of it. Here, for the low resolution of
location data, we only represent it by mapping the location to an infrastructure object. \\

In some cases, this rough description may not provide enough information for the query user if the infrastructure object is very large in space. For example, if it is on a highway the length of which can be twenty or thirty kilometers or on a relatively long city street, e.g., two or three kilometers, then the rough location representation with only object identifier seems to be not enough. To provide the location data in a more precise and meaningful way, we can do a second partition on each infrastructure object. The area covered by an infrastructure object can be partitioned into a set of pieces and $i_{loc_1}.\delta_1$ ($i_{loc_2}.\delta_1$) can be used to store the partition identifier. 
The partition methods are different for each kind of infrastructure depending on the data type used to represent the infrastructure objects. For the infrastructures and infrastructure objects defined in Section \ref{sec:infrastructure}, we propose the partition methods in the following. 

\begin{itemize}
 \item $I_{ptn}(\underline{mpptn})$ 

We believe it is enough to know which bus or train the traveler takes, while the relative location inside can be ignored. Thus, for this kind of object no partition is needed. 

 \item  $I_{indoor}(\underline{groom})$ 

We first briefly describe the data type $\underline{groom}$. Basically, it represents a room by a 2D area plus a value denoting the height above sea level. We use the type $\underline{region}$ to denote a 2D area and do a cell partition on it where each cell can be represented by a rectangle (region). So, the low resolution location description is composed of two parts: a room identifier $oid$ and a cell \textit{id}. The cell size depends on the application requirement. 

 \item  $I_{rbo}(\underline{region})$

We use the same method as for $I_{indoor}$ that a region is partitioned into a set of disjoint cells. 

 \item  $I_{rn}(\underline{line})$

As a line consists of a sequence of segments, each partition can correspond to a segment. The rough
location description first references to a line and second references to a segment in that line. 

 \item  $I_{fs}$ \hspace{0.2cm} No partition.
\end{itemize}

Although the second partition is better than only referencing to an object identifier, the location description is still at an imprecise level. Here, we just show that our model can represent the low resolution location in two levels, 
while whether the second partition is required is determined by the application. In the rest of the paper, we assume only the object identifier is used for rough location description. 

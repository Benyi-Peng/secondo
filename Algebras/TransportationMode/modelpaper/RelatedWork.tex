\section{Related Work}
\label{sec:relatedwork}
The related work includes the following three parts: \\

\textbf{Modeling Moving Objects} \\

In the database literature, there has been a large body of recent works 
\cite{PO+97,WX+98,GBE+00,FG+00,SXI01,WCKY01,HJ2003,SJ2003,BPT04,
DG2004,MR05,GA2006,KO07,PS07,JLSZ08,JLY209} 
on modeling moving objects, all of which only consider one specific environment and do not
discuss transportation modes. In addition, the location data is not represented in a 
multiresolution way. The closest to our work is \cite{BPT04,MR05,GA2006} 
where the infrastructure is considered for representing location data. 
A semantic model for trajectories is proposed in \cite{BPT04} where the background geographic information is considered. It is restricted to a specific
application domain where an algorithm is developed to map the positions of vehicles into a 
road network. Thus, the spatial aspect of trajectories is modeled in terms of a network, which consists of edges and nodes. A model for mobility pattern queries \cite{MR05} is 
proposed which relies on a \textit{discrete} view of spatio-temporal space. The movement of 
moving objects is based on a discrete representation of underlying space, 
called \textit{reference} space. The space is partitioned into a set of zones each of which is uniquely 
identified by a label. Afterwards, the location is represented by mapping it into zones and 
a trajectory is defined as a sequence of labels. A so-called \textit{route-oriented} model 
\cite{GA2006} is presented for moving objects in networks. It represents the road 
network by routes and junctions, and trajectories are integrated with road networks. 
As moving objects in a road network are located on roads and streets, 
the position can be represented by a route identifier 
and the relative position in that route, where a $line$ is used to describe the 
geometrical property of a route. However, the work above shares three common drawbacks: first, the \textit{infrastructure} there is
single, \textit{free space} in \cite{MR05} and \textit{road network} in \cite{BPT04,GA2006}, making the 
model not general and only feasible for one environment; 
second, the location representation is not in a multiresolution way where 
\cite{MR05} is in a coarse level, which is identified by a symbol corresponding to a zone 
in space (if a precise location query is requested, for example, what is the relative position in 
that zone, the model can not answer), \cite{BPT04} is also in a rough level (road segment)
and \cite{GA2006} only represents the precise location (without rough description); 
third, transportation modes are not handled. Besides, in \cite{MR05} the trajectory is represented in a discrete way (a sequence of timestamps) instead of continuous. \\

GPS is the dominant positioning technology for outdoor settings, but for the indoor 
environment new techniques are required. A graph model based approach \cite{JLY209} is 
proposed for indoor tracking moving objects using the technology $RFID$. 
They assume the $RFID$ readers are embedded in the indoor space in some known positions and 
the indoor space is partitioned into cells corresponding to vertices in the graph. 
An edge in the graph indicates the movement between cells which is detected by $partitioning$ 
$readers$. The raw trajectory is a sequence of RFID tags and from the raw trajectory they
construct and refine the trajectory. The goal is to improve 
the indoor tracking accuracy, which is different from the intention of this paper, modeling
moving objects. $Jensen$ $et$ $al.$ \cite{JLY109} present an index structure for moving objects 
in symbolic indoor space. The trajectory model is composed of records in the format 
$(oid,symbolicID,t)$ where $oid$ is the moving object identifier, $symbloicID$ is the 
identifier for a specific indoor space region and $T$ indicates time. They do not give the
data type representing a space region and the location there is imprecise. It is known
in which room the object is located, but the precise location inside is not represented. \\

\textbf{Transportation Modes and Semantic Trajectory} \\

A data model presented in \cite{BSWC09} gives the framework of a transportation
system which can provide a trip consisting of several transportation modes,
e.g., $Bus$, $Walk$, $Train$. It integrates moving objects and graph-based databases to
facilitate trip planning in urban transportation networks. They consider how to provide 
a path connecting an origin and destination including multiple transportation modes so that 
the $shortest\_path$ has more constraints and choices, like different motion modes, number of 
transfers, etc. They define a graph model where each vertex
corresponds to a place in a transportation network which has a name and a geometric  
representation, e.g., a point or a region. Each edge builds a connection between two vertices
and is associated with a kind of transportation mode attribute, for example, 
\textit{pedestrian}, \textit{auto}. Edges with different modes can be incident on the same 
vertices indicating that a transfer between different modes can happen. A trip is defined as a 
sequence of \textit{legs}, where each \textit{leg} represents a path with a kind of transportation mode. The model is for planning trips with various constraints, e.g., distance, duration, number
of transfers. It is different from our work that considers modeling moving objects. 
They do not focus on representing the location data for moving objects in various  
environments, but the constraints with the trip. It does not give the data type to represent 
the \textit{leg} and \textit{trip}, but describes them conceptually and abstractly. We focus on representing moving objects in different environments and manage the trips in a database system so that users can query them in a system. Besides, it does not include the \textit{indoor} environment. An interesting query 
called \textit{Isochrones} is considered in \cite{BGLPPT2008} which is to find the set of
all points on a road network so that a specific point of interest can be 
reached within a given time span. They do not focus on modeling moving objects and only consider 
two transportation modes, \textit{Walk} and \textit{Bus}. \\

The work in Microsoft's \textit{GeoLife} \cite{ZLWX08,ZCXM09} is different from ours. It focuses on 
discovering and inferring transportation modes from raw GPS trajectory data. The procedure 
comprises three phases. First, it partitions
a GPS trajectory into several segments of different transportation modes, while maintaining a 
segment of one mode as long as possible. It identifies a set of features being independent of the velocity. Second, these features are fed into a 
classification model and output the probability of each segment being different 
transportation modes. Third, a graph-based post-processing algorithm is proposed to further
improve the inference performance. We concentrate on representing moving objects
in various environments with different transportation modes instead of inferring the modes. The results of their work can be used as the raw data for our model. In addition, they only consider \textit{outdoor} movement because they infer based on GPS data where a GPS receiver will lose signal indoors. \\
		

Recently, semantic trajectory \cite{ABKMMV07,SPDMPV08,BKA09} receives more attention in the 
literature which complements the recording of moving position. It concerns enriching
trajectories with semantic annotations and allowing users to attach semantic data 
to specific parts of the trajectory. They define two parts called $stop$ and $move$ where a $stop$ is defined as a specified spatial location according to the application and a $move$ is the part of a 
trajectory delimited by two consecutive stops. The work does not propose a method for modeling 
moving objects but involves a data preprocessing model to add semantic information to 
moving objects trajectory before query processing. 
That is, the semantic data is not represented by the data model. 
Another difference is the semantic data there is an interesting 
place, e.g., a hotel or a touristic place, while ours is transportation modes. \\

\textbf{Multi-Scale Representation} \\

Paper \cite{MR04} studies on multi-scale classification of moving object trajectories. 
They propose a model that allows to classify trajectories with respect to a multi-scale
representation of a domain area, each scale level being a partition of the domain in zones. 
But the location represented there is only at the zone level, while the precise location
inside is unknown. Although it is a multi-scale representation, the location data is still
approximate. They focus on a specific query called \textit{trajectory pattern}, 
while we address modeling moving objects in various environments as well as transportation modes. A multi-scale model \cite{VMPR06} is proposed for vector maps which can identify the level of detail of entities forming in the map and allow for map storage and processing with an acceptable trade-off between cost and performance. It is different from our work that considers modeling moving objects. 

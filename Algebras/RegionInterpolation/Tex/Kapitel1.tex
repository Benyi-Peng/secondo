
\chapter[Einleitung \anmerkung{ca. 5 Seiten}]{Einleitung} \label{Kapitel1}


Heute sind Geodaten und die Verarbeitung von Geodaten in Geografischen Informationssystemen (GIS) allgegenwärtig. Geodaten helfen uns, sicher und schnell von einem Ort zu einem anderen zu kommen, sie unterstützen bei der schnellen Bewältigung von Krisen- und Katastropfenfällen und sind nützlich um globale Veränderungen anschaulich erfassen zu können.

Auch der Zuspruch, den Dienste wie etwa ,,Google-Earth'' mit ihren vielfältigen usergenerierten Inhalten, haben zeigt uns, dass GIS ein Thema ist, dass den Sprung in das Bewusstsein einer breiten Öffentlichkeit geschafft hat. 

Betrachtet man sich all diese Anwendungen, so fällt auf, dass Geodaten immer als statische Daten begriffen werden. Dies kann einerseits angesichts der Komplexität der Verarbeitung von dynamischen Geodaten nicht verwundern, wird aber andererseits vielen Daten mit geografischem Bezug nicht gerecht, denn alles bewegt sich. Zwar sind die großen Bewegungen der Erde im Universum oder der Kontinente zueinander in den meisten Anwendungsfällen nicht relevant, die Modellierung von Bussen durch Linien auf einem Stadtplan ist aber schon nur ein Notbehelf. Busse sind sichbewegende Punkte, und sollten, je nach Problemstellung, auch so modelliert werden. Wer eine gutgemachte Applikation sehen möchte, die genau diese Fragestellung löst sein der Besuch der Webseite \cite{swR} empfohlen.

Auch die Modellierung von anderen logistischen (etwa LKW oder Container, die bewegt werden), von meteorologischen Daten (Luftdruckgebiete, die über das Land ziehen) oder etwa die Ausbreitung von Entwaldungsgebieten im Amazonasgebiet sind einige Beispiele für Geodaten, die sich mit der Zeit bewegen, oder auch ihre Form ändern. Die Erfassung dieser  Daten erfolgt meistens in Form von Schnappschüssen, aus denen stetige Bewegungen berechnet werden müssen. 

Bei den logistischen Daten ist diese Berechnung noch recht einfach. Das Objekt, welches punktförmig modelliert werden kann, erfasst in festen Intervallen seine Position. Diese Erfassung kann zum Beispiel durch einen eingebauten GPS-Empfänger geschehen. Die Interpolation zwischen diesen Schnappschüssen kann linear erfolgen. 

Bei nicht punktförmigen Objekten hingegen sieht die Fragestellung deutlich schwieriger aus. Aber wofür braucht man den solche komplizierten sich bewegenden Daten? Betrachten wir zu dieser Fragestellung zwei Beispiele:

\subsection*{Flugzeugwartung}

In der Wartungsabteilung einer großen Flugliene will man die Wartungsintervalle der Flugzeuge minimieren. Zu diesem Zweck soll erfasst werden, welche Flugzeuge sich durch Unwetterzonen bewegt haben. Bei allen anderen Fliegern kann man dann nämlich das Intervall zur nächsten Wartung länger ansetzten. (Bitte beachten Sie, dass dieses Beispiel glücklicherweise rein fiktiv ist.)

Es stellt sich nun also die Frage, welches Flugzeug wann wo war, und wo sich gleichzeitig die Schlechtwetterzonen befanden. In dieser Fragestellung bewegen sich nicht nur die Flugzeuge, sondern auch die Wettergebiete.

Glücklicherweise kennt der Betriebsinformatiker das SECONDO-System, mit welchem sich diese Fragestellungen lösen lassen. SECONDO, welches unter \ref{SecondoEinfuehrung} näher beschrieben ist, ist ein erweiterbares Daten"-bank"-management-System, das unter anderem auch die Möglichkeit bietet sich bewegende Geodaten zu verwalten.

Die Erfassung der Flugzeugpositionen und die Berechnung von sich bewegenden Punkten aus diesen Daten, stellen wie bei dem Logistikbeispiel kein Problem dar.

Die Versuche, die Wetterdienste dazu zu bewegen, ihre Daten als bewegende Regionen in dem entsprechenden SECONDO-Format abzugeben, trugen erwartungsgemäß keine Früchte. Der Betriebsinformatiker kann nur die Polygone bekommen, die Lage und Form der Gebiete zu einigen wenigen Zeitpunkten beschreiben.

Die Aufgabe dieser Arbeit ist es jetzt dem armen Informatiker etwas unter die Arme zu greifen.

\begin{figure}
	\centering
	\includegraphics[scale=0.3]{Flugzeug.eps}
	\caption[Flugzeug zwischen Wettergebieten]{Beispiel für ein Flugzeug, welches sich zwischen sich bewegenden Wettergebieten bewegt.}
	\label{fig:Flugzeug}
\end{figure}


\subsection*{DETER-Daten}

Das die Abholzung der Regenwälder in Südamerika ein ernstes Problem darstellt sollte eigentlich jedem bekannt sein. So richtig ein Bild von dieser Entwicklung kann man sich aber nur, wenn man das Fortschreiten der Entwaldung auf einer kartographischen Darstellung verfolgt. (Es sei denn man hat die Möglichkeit sich selbst vor Ort zu begeben.) 

Die Daten der Entwaldung sind wieder eine kontinuierliche Bewegung. Die Daten, die durch die Auswertung von Satelietenphotos entstehen, können aber wiederum nur Momentaufnahmen dieser Entwicklung zeigen. Eindrucksvoller wäre es die Ausbreitungen auch kontinuierlich betrachten zu können. Auch ließen sich so Fragen näherungsweise beantworten, wie ,,Seit wann liegt der Punkt $x$ nicht mehr im Wald?''.

Die Aufgabe diese Daten in eine kontinuierliche Form zu überführen ist auch eine Aufgabe der vorliegenden Arbeit. (siehe Kapitel~\ref{Kapitel5})

\begin{figure}
	\centering
	\includegraphics[scale=0.4]{DeterEinf.eps}
	\caption[Ein Beispiel für Entwaldungsdaten]{Ein Beispiel der Entwaldungsdaten. Farbige Flächen sind abgeholzt, die Färbung gibt das Alter an, je roter desto jünger.}
	\label{fig:DeterEinfuehrung}
\end{figure}


\subsection*{Das weitere Vorgehen}

Zunächst beschreibe ich das SECONDO-System, das die Grundlage meiner Implementierung bildet, näher. Hierbei gehe ich natürlich besonders auf die Aspekte ein, welche ich zur Lösung des Problems benutzt habe (siehe~\ref{SecondoEinfuehrung}).

Dann betrachte ich die Literatur zu den Themen Interpolation und Ähnlichkeitsbetrachtung von Regionen und beschäftige mich mit drei Veröffentlichungen näher. Damit der Leser nicht alle drei Arbeiten quer lesen muss, fasse ich die Ergebnisse dieser Paper kurz zusammen (\ref{Tossebro}, \ref{AARR} und \ref{AFRWW}).

Im nächsten Kapitel gebe ich zunächst meine eigenen Überlegungen, wie man das Problem lösen könnte, wieder und bewerte die Ideen aus den vorgestellten Papern auf ihre Eignung zur Lösung des Problems (siehe~\ref{vorueberlegungen}).
 
Bevor es an die Implementierung gehen konnte, mussten die benötigten Algorithmen entwickelt werden (siehe~\ref{Algorithmen}). Dann gebe ich Details aus der Programmierung bekannt (\ref{klassen}, \ref{java} und \ref{rialgebra}).

In dem Kapitel~\ref{Kapitel4} gebe ich dann erste Erfahrungen aus der Arbeit mit dem neuen Werkzeug wieder und betreibe ich Feinjustierung von Parametern, welche sich nur durch Erfahrung vornehmen lassen.

Endlich beschäftige ich mich mit den DETER-Daten und erstelle ich die bewegenden Regionen aus diesen (siehe~\ref{Kapitel5}).

Zuletzt gebe ich einen Ausblick auf meine Arbeit, und gebe ich Anregungen, wie diese fortzuführen ist (siehe~\ref{Kapitel6}).
\anmerkung{Und weiter?}


%\anmerkung{Beschreibung der allgemeinen Problemstellung, man m"ochte Moving-Regions darstellen k"onnen z. B. f"ur die Deter Daten, Kurzbeschreibung dieser Daten, das f"uhr zu Secondo, und den Moving-Regions in Secondo. Die Schwierigkeiten bie der Erstellung von diesen, und die Struktur der Deter-Daten f"uhren zu dem Interpolations-Problem, dass im folgenden gel"oste werden soll.}
%\minitoc
%\newpage
%\section{Secondo \anmerkung{2 Seiten}}
%\subsection{Was ist Secondo}
%\anmerkung{Eine kuze Beschreibung, was Secondo ist, Die Entstehungsgeschichte und was Secondo so alles kann.}
%\section{Moving Regions \anmerkung{2 Seiten}}
%\anmerkung{Eine kurze Beschreibung, was Moving-Regions sind, und eine Beschreibung wozu diese gut sind. Eventuell k"onnte ich %hier das Flugzeugbeispiel benutzen}
%\subsection{Was sind Moving Regions?}

%\subsection{Wof"ur sind Moving Regions gut?}

%\section{Interpolation \anmerkung{1 Seite}}
%\anmerkung{Warum die Erzeugung von Moving-Regions kompliziert ist, und inwiefern diese Problemetik sich duch meine Arbeit verbesesrn wird.}
%\subsection{Was sind die Probleme bei der Erstellung von Moving Regions?}
%\subsection{Wie sieht die L"osung aus?}
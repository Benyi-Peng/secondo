
\chapter[Einleitung \anmerkung{ca. 5 Seiten}]{Einleitung\\
\normalsize{(kuze Einleitung, was ist Secondo, warum MovinRegions, warum Interpolation) \anmerkung{ca. 5 Seiten}}} \label{Kapitel1}
\anmerkung{Beschreibung der allgemeinen Problemstellung, man m"ochte Moving-Regions darstellen k"onnen z. B. f"ur die Deter Daten, Kurzbeschreibung dieser Daten, das f"uhr zu Secondo, und den Moving-Regions in Secondo. Die Schwierigkeiten bie der Erstellung von diesen, und die Struktur der Deter-Daten f"uhren zu dem Interpolations-Problem, dass im folgenden gel"oste dwerden soll.}


%\minitoc
%\newpage
%\section{Secondo \anmerkung{2 Seiten}}
%\subsection{Was ist Secondo}
%\anmerkung{Eine kuze Beschreibung, was Secondo ist, Die Entstehungsgeschichte und was Secondo so alles kann.}
%\section{Moving Regions \anmerkung{2 Seiten}}
%\anmerkung{Eine kurze Beschreibung, was Moving-Regions sind, und eine Beschreibung wozu diese gut sind. Eventuell k"onnte ich %hier das Flugzeugbeispiel benutzen}
%\subsection{Was sind Moving Regions?}

%\subsection{Wof"ur sind Moving Regions gut?}

%\section{Interpolation \anmerkung{1 Seite}}
%\anmerkung{Warum die Erzeugung von Moving-Regions kompliziert ist, und inwiefern diese Problemetik sich duch meine Arbeit verbesesrn wird.}
%\subsection{Was sind die Probleme bei der Erstellung von Moving Regions?}
%\subsection{Wie sieht die L"osung aus?}
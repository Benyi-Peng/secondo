
\chapter[Einleitung \anmerkung{ca. 5 Seiten}]{Einleitung\\
\normalsize{(kuze Einleitung, was ist Secondo, warum MovinRegions, warum Interpolation) \anmerkung{ca. 5 Seiten}}} \label{Kapitel1}
\anmerkung{Beschreibung der allgemeinen Problemstellung, man m"ochte Moving-Regions darstellen k"onnen z. B. f"ur die Deter Daten, Kurzbeschreibung dieser Daten, das f"uhr zu Secondo, und den Moving-Regions in Secondo. Die Schwierigkeiten bie der Erstellung von diesen, und die Struktur der Deter-Daten f"uhren zu dem Interpolations-Problem, dass im folgenden gel"oste werden soll.}

Heute sind Geodaten und die Verarbeitung von Geodaten in Geografischen Informationssystemen allgegenwärtig. Geodaten helfen uns, sicher und schnell von einem Ort zu einem anderen zu kommen, sie 

Auch der Zuspruch, den Dienste wie etwa ,,Google-Earth'' mit ihren vielfältigen usergenerierten Inhalten, haben zeigt uns, dass GIS ein Thema ist, dass den Sprung in das Bewußtsein einer breiten Öffentlichkeit geschaft hat. 

Betrachtet man sich all diese Anwendungen, so fällt auf, dass Geodaten immer als statische Daten begriffen werden. Dies kann einerseits angesichts der Komplexität der Verarbeitung von dynamischen Geodaten nicht verwundern, wird aber andererseits vielen Daten mit geografischem Bezug nicht gerecht, denn alles bewegt sich. Zwar sind die großen Bewegungen der Erde im Universum oder der Kontinente zueinander in den meisten Anwendungsfällen nicht relevant, die Modelierung von Bussen duch Linien auf einem Stadtplan ist aber schon nur ein Notbehelf. Busse sind sichbewegende Punkte, und sollten, je nachProblemstellung, auch so modelliert werden. Wer eine gutgemachte Applikation sehen möchte, die genau diese Fragestellung löst sein der Besuch der Webseite \cite{swR} empfohlen.

Auch die Modelierung von anderen logistischen (etwa LKW oder Container, die bewegt werden), von meterologischen Daten (Luftdruckgebiete, die über das Land ziehen) oder etwa die Ausbreitung von Gaswolken sind einige Beispiele für Geodaten, die sich mit der Zeit bewegen, oder auch ihre Form ändern. Die Erfassung dieser  Daten erfoglt meistens in Form von Schnappschüssen, aus denen stetige Bewegungen berechnet werden müssen. In den genannten Bespielen  Während d

%\minitoc
%\newpage
%\section{Secondo \anmerkung{2 Seiten}}
%\subsection{Was ist Secondo}
%\anmerkung{Eine kuze Beschreibung, was Secondo ist, Die Entstehungsgeschichte und was Secondo so alles kann.}
%\section{Moving Regions \anmerkung{2 Seiten}}
%\anmerkung{Eine kurze Beschreibung, was Moving-Regions sind, und eine Beschreibung wozu diese gut sind. Eventuell k"onnte ich %hier das Flugzeugbeispiel benutzen}
%\subsection{Was sind Moving Regions?}

%\subsection{Wof"ur sind Moving Regions gut?}

%\section{Interpolation \anmerkung{1 Seite}}
%\anmerkung{Warum die Erzeugung von Moving-Regions kompliziert ist, und inwiefern diese Problemetik sich duch meine Arbeit verbesesrn wird.}
%\subsection{Was sind die Probleme bei der Erstellung von Moving Regions?}
%\subsection{Wie sieht die L"osung aus?}
\documentclass[a4paper,10pt]{article}

\usepackage[left=2cm,right=2cm,top=2.5cm, bottom=2cm]{geometry}
\usepackage{hyperref}
\usepackage{amssymb}
\usepackage{url}
\usepackage{float}
\usepackage{longtable}
\usepackage{graphicx}

\pagestyle{headings}

\newcommand{\secondo}{\textsc{Secondo}}
\newcommand{\op}[1]{\textbf{#1}}
\newcommand{\var}[1]{\textsl{#1}}
\newcommand{\dt}[1]{\textsl{\underline{#1}}}
\newcommand{\file}[1]{\texttt{#1}}
%opening
\title{BerlinMOD and Network Test Instruction}
\author{Simone Jandt}
\date{Last Update: \today}
\begin{document}

\maketitle
\section{Preparation}
All scripts and files mentioned in the following should be placed in the \file{/secondo/bin}-directory to run properly.

The three data source files needed by the \file{BerlinMOD\_DataGenerator.SEC} can be found in 

\file{/secondo/Algebras/Network/DataSourceBerlinMOD}-directory\footnote{There are different versions of \file{street.data} file. The file \file{street.data.org} contains the original street data delieverd with the BerlinMOD Benchmark. The file \file{street.data.korr} contains a manual corrected version, because the original street source data has some failures causing problems in the map matching of the moving point data onto the network. You can find a more detailed description of the routefailure - problem in the translation section of \file{PaperBerlinMODAndNetwork}.}.
 
All scriptfiles for the BerlinMOD Benchmark and the Network BerlinMOD can be found in 
 
\file{/secondo/Algebras/Network/SecondoScripts}-directory.

\section{Generating Data}
For data generation you use first the \file{BerlinMOD\_DataGenerator.SEC} it can be customized by editing the parameter \file{SCALEFACTOR}. For a \file{SCALEFACTOR} of 1.0 it observes 2000 cars for 28 days. A \file{SCALEFACTOR} of 0.05 e.g. produces 447 cars observed for 1 week. 

After generating the source data you can use the script \file{BerlinMOD\_CreateObjects.SEC} to produce the source data and indexes for the original BerlinMOD Benchmark and the script \file{Network\_CreateObejcts.SEC} to produce the network translation of the BerlinMOD Benchmark data sources and the network indexes analogous to.

Alternatively the creation of the BerlinMOD Benchmark objects can be done from object files in \secondo{} nested list format generated by a option of the \file{BerlinMOD\_DataGenerator.SEC} script. But for bigger databases this takes longer than generating and creating the data with the scripts from the scratch.

\section{Benchmark Scripts}
\subsection{Original BerlinMOD}
The script \file{BerlinMOD\_OBA-Queries.SEC} contains the original BerlinMOD queries for the object based approach and the script \file{BerlinMOD\_TBA-Queries.SEC} contains the original BerlinMOD queries for the trip based approach of the BerlinMOD Benchmark.

Additional to the query scripts there are scripts deleting the saving the resulting query objects
 
\file{BerlinMOD\_OBA\_saveResults.SEC}, \file{BerlinMOD\_TBA\_saveResults.SEC} 

and deleting the resulting query objects 

\file{BerlinMOD\_OBA\_deleteResults.SEC}, \file{BerlinMOD\_TBA\_deleteResults.SEC}.


\subsection{Network BerlinMOD}
For the Network BerlinMOD we have the same structure. All \secondo{} Network Scripts start with \file{Network\_} follwed by OBA for the object based approach and by TBA for the trip based approach.

We have several \file{Queries}-scripts for experimental use. Which are explained in more detail below and \file{SaveResults} and \file{DeleteResults} files analogous to the save and delete scripts for the BerlinMOD Benchmark.

The query scripts filenames start all with \file{Network\_OBAQueries} respectively \file{Network\_TBAQueries}. After that a short version of the containing query is given:

\begin{itemize}
  \item \file{*OhneIndex.SEC} contains one network query for each of the 17 BerlinMOD queries that uses no index to solve the given problem.
  \item \file{*AllIndexVersions.SEC} contains one or more network queries for each of the 17 BerlinMOD queries using different indexes supporting faster query execution.
  \item \file{*AllIndexesMixed.SEC}  contains the same queries than \file{*AllIndexesVersions.SEC} but with in a other sortation, such that different queries for the same query object do not longer follow on each other to avoid cache effects.
  \item \file{*SelectedAPFast.SEC} was in the beginning a copy of \file{*AllIndexesMixed.SEC}. Used to select the query versions running fastest for each query on my working place pc.
  \item \file{*SelectedAPFast005.SEC} respectively \file{*SelectedAPFast020.SEC} collect the network query versions running fastet at scalefactor 0.05 respectively scalefactor 0.20 on my working place. 
\end{itemize}

\section{MONTree}
The script file \file{Network\_MONTreeQueries.SEC} contains some files used for the first MON-Tree tests.

\end{document}

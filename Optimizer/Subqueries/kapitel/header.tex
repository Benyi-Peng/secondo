%
% Headerdatei der Diplomarbeit
%
%

\documentclass[
		a4paper,
		12pt,
		twoside,
		openright,
		parskip,
		%draft,
		chapterprefix,%       Kapitel anschreiben als Kapitel
]{scrreprt}

\usepackage{moreverb}

%Deutsche Trennungen, Anf�hrungsstriche und mehr:
\usepackage[ngerman]{babel} 

%Eingabe von �,�,�,� erlauben
\usepackage[latin1]{inputenc}

\usepackage[babel,german=quotes]{csquotes}

%T1 Fonts benutzen
\usepackage[T1]{fontenc}
\usepackage{mathptmx}
\usepackage{lmodern}

%Zum Einbinden von Grafiken
\usepackage{graphicx}

% Silbentrennung
\usepackage{hyphenat}

%Ein Paket, das die Darstellung von "Text, wie er eingegeben wird"
%erlaubt: Also
%\begin{verbatim} \end{document}\end{verbatim} erzeugt die Ausgabe von
%\end{document} im Typewrites-Style und beendet nicht das Dokument.
\usepackage{verbatim}

%Source-Code printer for LaTeX
\usepackage{listings}

%Darstellung des Glossars einstellen
\usepackage[style=super, header=none, border=none, number=none, cols=2,
						toc=true]{glossary}

\makeglossary

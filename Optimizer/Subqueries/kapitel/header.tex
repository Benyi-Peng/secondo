%
% Headerdatei der Diplomarbeit
%
%

\documentclass[
		a4paper,
%		12pt,
%		twoside,
		openright,
		parskip,
%	draft,
%	chapterprefix,%       Kapitel anschreiben als Kapitel
]{scrreprt}

\usepackage{moreverb}

\usepackage{booktabs}

%Deutsche Trennungen, Anf�hrungsstriche und mehr:
\usepackage[ngerman]{babel} 

% zus�tzliche Silbentrennung
\hyphenation{Ent-schach-tel-ungs Im-ple-men-tie-rung HA-VING}

%Eingabe von �,�,�,� erlauben
\usepackage[latin1]{inputenc}

\usepackage[babel,german=quotes]{csquotes}

\usepackage{bibgerm}

\usepackage{color}

%\usepackage{lscape}

\usepackage{pgfpages}

\usepackage{pdfpages}

%\usepackage{lipsum}

%\usepackage{url}
\usepackage{hyperref}
\hypersetup{colorlinks=true, linkcolor=black, urlcolor=black, citecolor=black}


%T1 Fonts benutzen
\usepackage[T1]{fontenc}
\usepackage{amsmath}
\usepackage{lmodern}

%Zum Einbinden von Grafiken
\usepackage{graphicx}
\usepackage{tikz}
\usepackage{pgf}
\usepackage{pgfplots}
%\usetikzlibrary{tree}
\usetikzlibrary{arrows,decorations.pathmorphing,backgrounds,positioning,fit,calc,through}%\usetikzlibrary{datavisualization.formats.functions}

% Silbentrennung
%\usepackage{hyphenat}

%Ein Paket, das die Darstellung von "Text, wie er eingegeben wird"
%erlaubt: Also
%\begin{verbatim} \end{document}\end{verbatim} erzeugt die Ausgabe von
%\end{document} im Typewrites-Style und beendet nicht das Dokument.
\usepackage{verbatim}

%Source-Code printer for LaTeX
\usepackage{algorithmic}
\usepackage{algorithm}
\usepackage{listings}

\definecolor{ListingBackground}{rgb}{0.92,0.92,0.92}
%\definecolor{ListingBackground}{rgb}{1,1,1}

\lstset{%
  keywordstyle=\bfseries,
	language=SQL,     % Sprache des Quellcodes ist SQL	
	morekeywords={,groupby,orderby,theInstant,year_of,month_of,day_of,aggregate,ifthenelse,instant,}
	mathescape=true,	  % 
	captionpos=b,
	columns=fullflexible,
	escapeinside=��,    % Escapeumgebung f�r LaTeX
	numbersep=5pt,           % 5pt Abstand zum Quellcode
	numberstyle=\tiny,       % Zeichengr�sse 'tiny' f�r die Nummern.
	breaklines=true,         % Zeilen umbrechen wenn notwendig.
	breakautoindent=true,    % Nach dem Zeilenumbruch Zeile einr�cken.
	postbreak=\space,        % Bei Leerzeichen umbrechen.
	tabsize=2,               % Tabulatorgr�sse 2
	basicstyle=\ttfamily\footnotesize, % Nichtproportionale Schrift, klein f�r den Quellcode
	showspaces=false,        % Leerzeichen nicht anzeigen.
	showstringspaces=false,  % Leerzeichen auch in Strings ('') nicht anzeigen.
	extendedchars=true,      % Alle Zeichen vom Latin1 Zeichensatz anzeigen.
	backgroundcolor=\color{ListingBackground}} % Hintergrundfarbe des Quellcodes setzen.

\usepackage{multirow}


\def\MM#1{\ifmmode#1\else\mbox{$#1$}\fi}

\newcommand{\tc}[1]{\MM{\underline{\smash{\mathit{#1}}}}}
\DeclareFixedFont{\secop}{T1}{pcr}{b}{it}{8}

% define a lstlanguage for secondo queries and scripts
% emph[1] are operators
% emph[2] are types
\lstdefinelanguage{secondoscript}{keywords={let, update, restore, derive, save, database,
                                            create, open, query, list algebras },
                                  morekeywords={const, value, desc, asc, group, TRUE, FALSE, fun}
                                  sensitive=true,
                                  string=[b]{"},
                                  comment=[l]{\#},
                                  emph=[1]{ ANY, ANY2, ELEMENT, ELEMENT2, GROUP, PSTREAM1, PSTREAM2, 
        STREAMELEM, STREAMELEM2, TUPLE, TUPLE2, abs, addcounter, 
        addid, addtupleid, adjacent, aggregate, aggregateB, aggregateS, 
        always, and, approximate, area, at, atinstant, atmax, atmin, atperiods, 
        atpoint, atposition, attr, attrsize, avg, bbox, bbox2d, before, between, 
        bool2int, boundary, box2d, box3d, breakpoints, bulkloadrtree, cancel, category, 
        ceil, char, circle, colordist, commonborder, components, concat, concatS, 
        connectedcomponents, constgraph, constgraphpoints, consume, contains,
        cost, count, count2, create\_duration, create\_instant, createbtree, 
        createdeleterel, createinsertrel, creatertree, createupdaterel, crossings, 
        cumulate, cut, day\_of, deftime, deletebtree, deletebyid, deletedirect, 
        deletedirectsave, deletertree, deletesearch, deletesearchsave, derivable, derivable\_new, 
        derivative, derivative\_new, dice, direction, display, distance, distribute, disturb, div, 
        dumpstream, duration2real, echo, edges, equal, equals, equalway, exactmatch, exactmatchS, 
        export, extattrsize, extdeftime, extend, extendstream, extract, exttuplesize, feed, filename, 
        filter, final, flipleft, floor, get, get\_duration, gettuples, gettuples2, gettuplesdbl, 
        getx, gety, gps, groupby, hashjoin, hashvalue, head, height, hour\_of, ifthenelse, ininterior, 
        initial, insert, insertbtree, insertrtree, insertsave, inserttuple, inserttuplesave, inside, 
        inst, instant2real, int2bool, int2real, integrate, intersection, intersection\_new, intersects, 
        intersects\_new, intstream, isempty, isgrayscale, isportrait, junctions, key, keywords, krdup, 
        ldistance, leapyear, leftrange, leftrangeS, length, like, line2region, linearize, linearize2, 
        locations, log, loop, loopa, loopb, loopjoin, loopjoinrel, loopsel, loopselect, loopselecta, 
        loopselectb, loopswitch, loopswitcha, loopswitchb, makearray, makearrayN, makemvalue, makepoint, 
        max, maxD, maxDuration, maxInstant, maxdegree, maximum, mbool2mint, mconsume, mdirection, 
        memshuffle, merge, mergediff, mergejoin, mergesec, mergeunion, millisecond\_of, min, minD, 
        minDuration, minInstant, mindegree, minimum, minus, minus\_new, minute\_of, mirror, mod, 
        month\_of, move, namedtransformstream, never, no\_components, no\_entries, no\_nodes, 
        no\_segments, nodes, nonequal, not, now, num2string, onborder, or, overlaps, p\_intersects, 
        partjoin, partjoinselect, partjoinswitch, partof, passes, pcreate, pcreate2, pdelete, 
        perimeter, periods2mint, pfeed, picturedate, pjoin1, pjoin2, placenodes, point2d, polylines, 
        pos, present, printintstream, printstream, product, project, projectextend, projectextendstream, 
        projecttransformstream, pshow, puse, put, queryrect2d, randint, randmax, randseed, range, 
        rangeS, rangevalues, rdup, real2int, realm, realstream, rect2region, rectangle2, rectangle3, 
        rectangle4, rectangle8, rectproject, reduce, relcount, relcount2, remove, rename, reverse, 
        rightrange, rightrangeS, rng\_GeneratorMaxRand, rng\_GeneratorMinRand, rng\_GeneratorName, 
        rng\_NoGenerators, rng\_binomial, rng\_exponential, rng\_flat, rng\_gaussian, rng\_geometric, 
        rng\_getMax, rng\_getMin, rng\_getSeed, rng\_getType, rng\_init, rng\_int, rng\_intN, 
        rng\_poisson, rng\_real, rng\_realpos, rng\_setSeed, rootattrsize, roottuplesize, 
        rough\_center, round, routes, sample, samplempoint, saveto, scale, second\_of, sections, 
        segments, sentences, seqinit, seqnext, setoption, setunitofdistance, setunitoftime, 
        shortestpath, shuffle, sim\_create\_trip, sim\_fillup\_mpoint, sim\_print\_params, 
        sim\_set\_dest\_params, sim\_set\_event\_params, sim\_set\_rng, sim\_trips, simpleequals, 
        simplify, single, size, sizecounters, sometimes, sort, sortarray, sortby, sortmergejoin, 
        source, spatialjoin, speed, speed\_new, sqrt, starts, subline, substr, sum, summarize, 
        symmjoin, symmproduct, symmproductextend, target, theInstant, theRange, the\_ivalue, 
        the\_mvalue, the\_unit, thedate, theday, thehour, theminute, themonth, thenetwork, theperiod, 
        thesecond, thevertex, theyear, tie, today, touchpoints, trajectory, transformstream, 
        translate, translateappend, translateappendS, treeheight, tupleid, tuplesize, 
        uint2ureal, union, union\_new, units, updatebtree, updatebyid, updatedirect, updatedirectsave, 
        updatertree, updatesearch, updatesearchsave, upper, use, use2, val, velocity, velocity\_new, 
        vertextrajectory, vertices, weekday\_of, width, windowclippingin, windowclippingout, 
        windowintersects, windowintersectsS, within, year\_of, zero },
                                  emph=[2]{ array, binfile, bool, btree, date, duration, 
                                            edge, gpoint, graph, histogram, 
        ibool, iint, instant, int, intimeregion, ipoint, ireal, istring, line, map, 
        mbool, mint, movingregion, mpoint, mreal, mrel, mstring, mtuple, network, 
        path, periods, picture, point, points, polygon, ptuple, rbool, real, rect, 
        rect3, rect4, rect8, region, rel, rint, rreal, rstring, rtree, rtree3, rtree4, 
        rtree8, string, text, tid, trel, tuple, ubool, uint, upoint, ureal, uregion, 
        ustring, vertex, xpoint, xrectangle}
                                  }
% define a style for formatting secondo commands
\definecolor{gray}{rgb}{0.5, 0.5, 0.5}
\lstdefinestyle{secondostyle}{basicstyle=\rm\small,
                              keywordstyle=\bfseries,
                              commentstyle=\color{gray}\it, 
                              stringstyle=\texttt,
                              showspaces=false,
                              showstringspaces=false,
                              emphstyle=[1]\secop,
                              emphstyle=[2]\tc,
                              upquote=false,
                              language=secondoscript}


\usepackage{alltt}

%Darstellung des Glossars einstellen
\usepackage[style=super, header=none, border=none, number=none, cols=2,
						toc=true]{glossary}
						

\newcommand{\op}[1]{\operatorname{\textbf{#1}}}
\newcommand{\secondo}[1]{\texttt{#1}}					
\newcommand{\sql}[1]{\textsc{#1}}
\newcommand{\algebra}[1]{\emph{#1}}
\newcommand{\prolog}[1]{\emph{#1}}
\newcommand{\optional}[1]{\emph{#1}}

\makeglossary

%
% EOF
%
%

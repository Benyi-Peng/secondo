%
% Einleitung
%
%

\chapter{Einleitung}\label{chp:Einleitung}
\pagenumbering{arabic}


\section{Problembeschreibung}\label{sct:Problembeschreibung}
SQL ist eine m�chtige Abfragesprache, mit deren Hilfe sich viele Fragestellungen, die an relationale Daten gestellt werden k�nnen, formulieren lassen. Da SQL nach dem Vorbild der englischen Sprache entworfen wurde, erleichtert die Syntax die formale Darstellung in Umgangssprache vorliegender Informationsanforderungen. Einer der gro�en Vorteile von SQL ist die M�glichkeit Abfragen geschachtelt zu formulieren. So l�sst sich die Fragestellung \enquote{Gib mir alle Lieferanten, deren Umsatz dem maximalen Umsatz aller Lieferanten entspricht} unter der Annahme, dass es eine Relation \enquote{Lieferanten} mit dem Schema (Name, Umsatz) gibt, formulieren als \sql{select * from lieferanten where umsatz = (select max(umsatz) from lieferanten)}. Die in einem Nebensatz formulierte Bedingung \enquote{deren Umsatz dem maximalen Umsatz aller Lieferanten entspricht} kann hier direkt als Abfrage formuliert werden. Der \textsc{Secondo}-Optimierer versteht bereits einen SQL-Dialekt, kann aber noch nicht mit geschachtelten Abfragen umgehen. Ziel dieser Arbeit war es, den Optimierer in die Lage zu versetzen, geschachtelte Abfragen �bersetzen zu k�nnen und eine m�glichst effiziente Ausf�hrungsstrategie hierf�r auszuw�hlen.

\section{Aufbau der Arbeit}\label{sct:Aufbau der Arbeit}
Die Arbeit besteht aus vier Teilen.

Im ersten Teil (Kapitel \ref{chp:Review}, ab Seite \pageref{chp:Review}) werden die technologischen und theoretischen Grundlagen dargelegt, mit denen sich geschachtelte Abfragen optimieren lassen. 

Der zweite Teil widmet sich der Problemanalyse und der ausf�hrlichen Darstellung der ausgew�hlten Algorithmen und wird in Kapitel \ref{chp:Entwurf} ab Seite \pageref{chp:Entwurf} dargestellt.

Die Implementierung und technische Umsetzung der Algorithmen wird im dritten Teil, Kapitel \ref{chp:Implementierung} Seite \pageref{chp:Implementierung} geschildert.

Im vierten Teil werden zwei m�gliche Ausf�hrungsstrategien f�r die �bersetzung geschachtelter Abfragen anhand von geschachtelten Abfragen aus dem TPC-D Benchmark quantitativ und strukturell verglichen.

%
% EOF
%
%
%
% Kapitel Implementierung
%
%

\chapter{Implementierung}
\section{Ein Outerjoin-Operator f�r SECONDO}
\begin{itemize}
	\item Beschreibung Sortmergejoin
	\subitem Zwei Tupelbuffer
	\subitem sortieren
	\subitem immer mit dem kleineren Tupel arbeiten und vergleichen
	\subitem falls Treffer, merge Schritt
	\subitem sonst Tupel verwerfen und mit n�chstem Tupel weiterarbeiten
\end{itemize}
Basis ist sortmergejoin, jedes nicht \enquote{matchende} Tupel wird mit NULL-Werten aufgef�llt, nur Equi-Outerjoin\\
Ausblick: allgemeiner Outerjoin auf Basis von Symmjoin, zwei Hash-Tabelle mit Tupel-Ids f�r alle gematchten Tupel, Tupelbuffer durchlaufen und ein mit NULL-Werten aufgef�lltes Tupel erzeugen f�r alle nicht gematchten Tupel
\section{Umschreiben quantifizierter Pr�dikate}
Pr�dikate mit den Operatoren \textsc{EXISTS, NOT EXISTS, ANY} und \textsc{ALL} werden in �quivalente Pr�dikate mit Aggregationen �berf�hrt.

\section{Entschachtelung}
Implementierung des allgemeinen Entschachtelungsalgorithmus nach Kim/Ganski,Wong. Rekursive Entschachtelung von Abfragen. Dabei werden die oben beschriebenen Schachtelungstypen unterschieden. Je nach Schachtelungstyp wird der entsprechende Entschachtelungs-Algorithmus angewendet. \enquote{Tempor�re Relationen} werden f�r die Entschachtelung korrelierter Abfragen mit Aggregationsfunktion ben�tigt. Sie werden beim Schlie�en der Datenbank gel�scht.

\section{Schema-Lookup}
\begin{itemize}
	\item Selektionsattribute der Subquery
	\item Innere Tabellen
	\item Pr�dikate
	\subitem nach dem Lookup eines Pr�dikats immer die Selektivit�t ermitteln, um die globalen Datenstrukturen (pog etc.) nicht w�hrend der Plan-Ermittlung durch Subqueries zerst�ren
\end{itemize}

\section{Ermittlung des Ausf�hrungsplans}
\begin{itemize}
	\item korrelierte Pr�dikate der Subquery entfernen
	\item Plan f�r Subquery ohne korrelierte Pr�dikate ermitteln. Unterabfragen ohne korrelierten Pr�dikate werden durch die Standardmechanismen des \textsc{Secondo}-Optimierers �bersetzt, um einen m�glichst effizienten Ausf�hrungsplan zu ermitteln. 
	\item korrelierte Pr�dikate mit aufsteigender Selektivit�t an Plan \enquote{anflanschen}. 
\end{itemize}

\section{�bersetzung des Plans in ausf�hrbare Syntax}
\subsection{IN Operator}
�bersetzung mit den \textsc{Secondo} Operatoren \secondo{in}, \secondo{collect\_set} und \secondo{projecttransformstream}. 
Der Operator \secondo{projecttransformstream} hat die Signatur 
\secondo{$stream(tuple((a_1 t_1)\cdots(a_n t_n)))\times a_n$ ---> $(stream t_n)$} 
und wandelt einen Tupel-Strom in einen Strom von Werten um. Hierbei wird das Selektions-Attribute der geschachtelten Abfrage als zu w�hlender Parameter mitgegeben. Der Strom wird dann mit dem Operator \secondo{collect\_set} in eine Menge umgewandelt und f�r jedes Tupel des �u�eren Stroms wird mit \secondo{in} gepr�ft, ob der entsprechende Attributwert enthalten ist. 

\begin{lstlisting}
�$A_i$� in (select �$A_j$� from �$R_n$�)
\end{lstlisting}

wird in den \textsc{Secondo}-Ausdruck 

\secondo{filter[.$A_i$ in $R_n$ feed projecttransformstream[$A_j$] collect\_set]} �bersetzt. 
\subsection{Using Parameter functions}
\begin{itemize}
	\item Mit einem Parameter (normalerweise Filter)
	
	\item mit zwei Parametern f�r symmjoin (akzeptiert allgemeine Funktion)
\end{itemize}



%
% EOF
%
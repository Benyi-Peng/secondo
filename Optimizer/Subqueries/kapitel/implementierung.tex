%
% Kapitel Implementierung
%
%

\chapter{Implementierung}
\section{Ein Outerjoin-Operator f�r SECONDO}
Basis ist sortmergejoin, jedes nicht \enquote{matchende} Tupel wird mit NULL-Werten aufgef�llt, nur Equi-Outerjoin\\
Ausblick: allgemeiner Outerjoin auf Basis von Symmjoin, zwei Hash-Tabelle mit Tupel-Ids f�r alle gematchten Tupel, Tupelbuffer durchlaufen und ein mit NULL-Werten aufgef�lltes Tupel erzeugen f�r alle nicht gematchten Tupel
\section{Umschreiben quantifizierter Pr�dikate}
Pr�dikate mit den Operatoren \textbf{EXISTS, NOT EXISTS, ANY} und \textbf{ALL} werden in �quivalente Pr�dikate mit Aggregationen �berf�hrt.

\section{Entschachtelung}
Implementierung des allgemeinen Entschachtelungsalgorithmus nach Kim/Ganski,Wong. Rekursive Entschachtelung von Abfragen. Dabei werden die oben beschriebenen Schachtelungstypen unterschieden. Je nach Schachtelungstyp wird der entsprechende Entschachtelungs-Algorithmus angewendet. \enquote{Tempor�re Relationen} werden f�r die Entschachtelung korrelierter Abfragen mit Aggregationsfunktion ben�tigt. Sie werden beim Schlie�en der Datenbank gel�scht.

\section{Schema-Lookup}
\begin{itemize}
	\item Selektionsattribute der Subquery
	\item Innere Tabellen
	\item Pr�dikate
	\subitem nach dem Lookup eines Pr�dikats immer die Selektivit�t ermitteln, um die globalen Datenstrukturen (pog etc.) nicht w�hrend der Plan-Ermittlung durch Subqueries zerst�ren
\end{itemize}

\section{Ermittlung des Ausf�hrungsplans}
\begin{itemize}
	\item korrelierte Pr�dikate der Subquery entfernen
	\item Plan f�r Subquery ohne korrlierte Pr�dikate ermitteln
	\item korrelierte Pr�dikate mit aufsteigender Selektivit�t an Plan \enquote{anflanschen}
\end{itemize}

\section{�bersetzung des Plans in ausf�hrbare Syntax}
\subsection{IN Operator}
�bersetzung mit den Operatoren \begin{verbatim}collect_set und transformstream\end{verbatim}
\subsection{Using Parameter functions}
\begin{itemize}
	\item Mit einem Parameter (normalerweise Filter)
	\item mit zwei Parametern f�r symmjoin (akzeptiert allgemeine Funktion)
\end{itemize}



%
% EOF
%